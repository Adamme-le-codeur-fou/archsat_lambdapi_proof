% $Id$

\section{Rigid unit superposistion}

\EnableBpAbbreviations{}

There are many ways of integrating equational reasoning in
Tableaux methods~\cite{brand1975proving,letz2002integration,backeman2015theorem,degtyarev1996you}.
We propose here an approach based on superposition with rigid variables,
as in previous work by Degtyarev and Voronkov~\cite{Degtyarev}, but with
significant differences.
The first difference is that to avoid constraint solving, we do not use
basic superposition nor constraints.
The second difference is a {\em merging} rule that factors together
intermediate (dis)equations that are alpha-equivalent:
with multiple instances of the quantified formulas, it becomes important
not to duplicate work.


We will write $s \approx t | \Sigma$ (resp. $s \not\approx t | \Sigma$),
the unitary clause that contains exactly one equation (resp.~inequation)
under hypothesis $\Sigma$ (which is a set of subtitutions).
We will write $\emptyset | \Sigma$ for the empty clause under hypothesis $\Sigma$.

We want to express a notion of compatibility on substitutions, using a
partial ordering $\leq$, such that $\sigma \leq \sigma'$ means that $\sigma$ is
less general thatn $\sigma'$. We will first define a few notions.


Considering a substitution as a function from variables to terms, we can define
the domain of a substitution $\sigma$ as the set of variables which have a non-trivial
binding in $\sigma$.\footnote{a trivial binding maps a variable to itself.}
The co-domain of a substitution is the set of variables occuring in terms in the image of
the domain of the substitution.
In all the following, we will consider idempotent substitutions, i.e.~substitution for which
the domain and co-domain have an empty intersection.


We define the application (or composition~?) of substitutions $\sigma \circ \sigma'$.
The composition is well-defined iff the domains of $\sigma$ and $\sigma'$ have no
intersection. We then have
$\sigma \circ \sigma' = \left\{ x \mapsto \sigma'(\sigma(x)) | x \in \text{domain}(\sigma) \right\}$
This definition naturally extends to sets of substitutions:
$\Sigma \circ \sigma' = \left\{ \sigma \circ \sigma' | \sigma \in \Sigma \right\}$


We then say that $\sigma \leq \sigma'$ iff $\exists \sigma'' \sigma \circ \sigma'' = \sigma'$.
We extend that notion to set of substitutions:
$\Sigma \leq \Sigma'$ iff $\forall \sigma' \in \Sigma'. \exists \sigma \in \Sigma. \sigma \leq \sigma'$.

Then, the merging of two substitutions $\sigma \uparrow \sigma'$ as the supremum of $\{\sigma,\sigma'\}$
for the order $\leq$, if it exists. This definition naturally extends to sets:
$\Sigma \uparrow \Sigma' = \left\{ \sigma \uparrow \sigma' | \sigma \in \Sigma, \sigma' \in \Sigma' \right\}$


The idea is that for every equality (resp.~inequality) $e = f$ (resp.~$e \neg f$) containing meta-variables,
we add the clause $\sigma(e) = \sigma(f) | \{ \sigma \}$ to the superposistion state, with
$\sigma$ a substitution that maps each rigid meta-variable to a fresh variable.


\begin{figure}
  \begin{center}

    % ER
    \AXC{$s \not\approx t |\Sigma$}
    \LL{ER}
    \RL{if $\Sigma' = \Sigma \circ \text{mgu}(s, t)$}
    \UIC{$\emptyset | \Sigma'$}
    \DP{} \\[12pt]

    % SN
    \AXC{$s \approx t | \Sigma$}
    \AXC{$u \not\approx v | \Sigma'$}
    \LL{SN}
    \BIC{$\sigma''(u[p \leftarrow t] \not\approx v) | (\Sigma \circ \sigma'') \uparrow (\Sigma' \circ \sigma'')$}
    \DP{}
    $\begin{matrix}
        \sigma'' = \text{mgu}(u_{|p}, s) \\
        \sigma''(s) \not\leq \sigma''(t) \\
        \sigma''(u) \not\leq \sigma''(v) \\
        u_{|p} \not\in V \\
    \end{matrix}$ \\[12pt]

    % SP
    \AXC{$s \approx t | \Sigma$}
    \AXC{$u \approx v | \Sigma'$}
    \LL{SP}
    \BIC{$\sigma''(u[p \leftarrow t] \approx v) | (\Sigma \circ \sigma'') \uparrow (\Sigma' \circ \sigma'')$}
    \DP{}
    $\begin{matrix}
        \sigma'' = \text{mgu}(u_{|p}, s) \\
        \sigma''(s) \not\leq \sigma''(t) \\
        \sigma''(u) \not\leq \sigma''(v) \\
        u_{|p} \not\in V \\
    \end{matrix}$ \\[12pt]

    % TD1
    \AXC{$s \approx s | \_ $}
    \LL{TD1}
    \doubleLine{}
    \UIC{.}
    \DP{} \\[12pt]

    % TD2
    \AXC{$s \approx t | \emptyset$}
    \LL{TD2}
    \doubleLine{}
    \UIC{.}
    \DP{} \\[12pt]

    % PS
    %\AXC{$s \approx t |\sigma$}
    %\AXC{$u[p \leftarrow \sigma''(s)] \not\approx u[p \leftarrow \sigma''(t)] | \sigma' $}
    %\LL{PS}
    %\doubleLine{}
    %\BIC{$s \approx t | \sigma$ \qquad $\emptyset | \sigma \cup \sigma' \cup \sigma''$}
    %\DP{} \\[12pt]

    % NS
    %\AXC{$s \not\approx t | \sigma$}
    %\AXC{$\sigma''(s \approx t) | \sigma'$}
    %\LL{NS}
    %\doubleLine{}
    %\BIC{$s \not\approx t | \sigma$ \qquad $\emptyset | \sigma \cup \sigma' \cup \sigma''$}
    %\DP{} \\[12pt]

    % ME
    \AXC{$\rho(u) \approx \rho(v) | \Sigma$}
    \AXC{$u \approx v | \Sigma'$}
    \LL{ME}
    \RL{$\rho \text{ is a variable renaming}$}
    \doubleLine{}
    \BIC{$\rho(u) \approx \rho(v) | \Sigma \cup (\Sigma' \circ \rho)$}
    \DP{} \\[12pt]

    % ES
    \AXC{$s \approx t | \Sigma$}
    \AXC{$u[p \leftarrow s] \approx u[p \leftarrow t] | \Sigma' \cup \{ \sigma'\}$}
    \LL{ES}
    \RL{$\exists \sigma \in \Sigma. \sigma \leq \sigma'$}
    \doubleLine{}
    \BIC{$s\approx t | \Sigma \qquad u[p\leftarrow s] \approx u[p \leftarrow t] | \Sigma'$}
    \DP{} \\[12pt]

    % RP
    \AXC{$s \approx t | \Sigma$}
    \AXC{$u \approx v | \Sigma'$}
    \LL{RP}
    \doubleLine{}
    \BIC{$s \approx t | \Sigma$ \qquad $u[p \leftarrow t] \approx v | \Sigma'$}
    \DP{}
    $\begin{matrix}
      u_{|p} = s \\
      s \geq t \\
      \Sigma \leq \Sigma'\\
      \left\{ \begin{matrix} u \not\geq v \\ \text{or}\; p \neq \lambda\end{matrix}\right. \\
    \end{matrix}$ \\[12pt]

    % RN
    \AXC{$s \approx t | \emptyset$}
    \AXC{$u \not\approx v | \sigma$}
    \LL{RN}
    \doubleLine{}
    \BIC{$s \approx t | \emptyset$ \qquad $u[p \leftarrow t] \not\approx | \sigma$}
    \DP{}
    $\begin{matrix}
      u_{|p} = s \\
      s \geq t \\
      \Sigma \leq \Sigma'\\
    \end{matrix}$

  \end{center}
\end{figure}

