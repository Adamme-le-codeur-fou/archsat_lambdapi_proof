% $Id$

\section{Introduction}

These days, Boolean satisfiability (SAT) solvers are undeniably quite reliable
tools despite the worst-case exponential run time of all known
algorithms. Modern SAT solvers provide a ``black-box'' procedure that can often
solve hard structured problems with over a million variables and several million
constraints. These tools are therefore used in many applications outside the
scope of knowledge representation and reasoning, which nonetheless was their
initial domain of application. The underlying representational formalism of SAT
solvers is propositional logic, and even if a given problem is not a pure
propositional problem, it is not unusual to encode it so that it can be viewed
as a propositional reasoning task. This encoding may be complex and possibly
costly, but the effectiveness of SAT solvers generally compensates for the cost
of this encoding.

This growing interest for SAT solvers has steered them toward a more first order
approach, with the possibility to deal with first order literals and reason
modulo theories. This resulted in the design of Satisfiability Modulo Theories
(SMT) solvers, which are built on top of SAT solvers. Over the last past few
years, SMT solvers have appeared as very efficient tools to reason over some
well identified theories (equality, uninterpreted functions, linear arithmetic,
arrays, etc.), and have allowed us to drag SAT solving toward first order logic.
Although modern SMT solvers supports first order logic, most of them use
heuristic quantifier instantiation for incorporating quantifier reasoning with
ground decision procedures. This mechanism is relatively effective in some cases
in practice, but it is not refutationally complete for first order logic. Hints
(triggers) are usually required, and it is sensitive to the syntactic structure
of the formula, so that it fails to prove formulas that can be easily discharged
by provers based on more traditional first order proof search methods (tableaux,
resolution, etc.).

In this paper, we propose to overcome this problem of completeness for first
logic in SMT solving by combining a SAT solver core with tableau calculus. The
tableau calculus is introduced as a regular theory of SMT solving, and the rules
are used to unfold propositional content into clauses while atomic formulas are
handled using satisfiability decision procedures as usually done in SMT solvers.
In order to manage quantified first order formulas, we use metavariables and
perform rigid unification modulo equalities, for which we introduce an algorithm
based on superposition, but where all clauses contain a single atomic formula.

Our approach provides several advantages compared to usual SMT solving and first
order proof search methods. First, we benefit from the efficiency of a SAT
solver core together with a complete method of instantiation (when a
propositional model is found, we try to find a conflict between two literals by
unification). Second, it should be noted that our approach requires no change in
the architecture of the SMT solver. The tableau calculus is seen as a regular
theory, which can be combined with other theories in particular. Finally, no
preliminary Skolemization and Conjunctive Normal Form (CNF) transformation is
required over the initial formula. This transformation is performed lazily by
applying the tableau rules progressively when a literal is propagated or
decided. This makes the production of genuine output proofs easier, contrary to
the approach of SMT solvers, where the Skolemization/CNF translation is realized
at the beginning and especially externalized with respect to the proof search.

Our proposal combining SAT solving and tableau calculus has been implemented and
the corresponding tool is called \archsat{}. The implementation relies on the
\msat{} library, which derives from \altergoz{}, and which is a generic library
that allows us to easily create automated deduction tools based on SAT solvers.

\todo{Introduce benchmarks and experimental results.}

The paper is organized as follows: in Sec.~\ref{sec:smt}, we recall the classic
architecture of SMT solvers and we introduce the tableau theory; we then
describe, in Sec.~\ref{sec:rusuper}, our mechanism of equational reasoning by
means of rigid unit superposition; finally, in Sec.~\ref{sec:bench}, we present
some experimental results obtained by running our implementation over the TPTP
library, and also provide related work in Sec.~\ref{sec:rwork}.
