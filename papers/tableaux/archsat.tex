% $Id$

\documentclass{llncs}

\usepackage[T1]{fontenc}
\usepackage[usenames,dvipsnames,svgnames,table]{xcolor}
\usepackage{makeidx}
\usepackage{bussproofs}
\usepackage{amsmath,amsfonts,amssymb}
\usepackage{stmaryrd}
\usepackage{tabularx}
\usepackage{pgf,tikz}
\usepackage{url}
\usepackage{float}
\floatstyle{boxed}
\restylefloat{figure}

% $Id$

\def\ai{\textsf{Aesthetic~Integration}}
\def\alloy{\textsf{Alloy}}
\def\altergo{\textsf{Alt-Ergo}}
\def\altergoz{\textsf{Alt-Ergo~Zero}}
\def\archsat{\textsf{ArchSAT}}
\def\atelierb{\textsf{Atelier~B}}
\def\bbook{\textsf{B}-Book}
\def\bmth{\textsf{B}}
\def\bware{\textsf{BWare}}
\def\cvc{\textsf{CVC4}}
\def\e{\textsf{E}}
\def\ens{\textsf{ENS Paris-Saclay}}
\def\git{\textsf{Git}}
\def\inria{\textsf{Inria}}
\def\intel{\textsf{Intel~Xeon~E5-1650~v3}}
\def\iproverm{\textsf{iProver~Modulo}}
\def\lirmm{\textsf{LIRMM}}
\def\lsv{\textsf{LSV}}
\def\msat{\textsf{mSAT}}
\def\ocaml{\textsf{OCaml}}
\def\princess{\textsf{Princess}}
\def\um{\textsf{Université de Montpellier}}
\def\vdm{\textsf{VDM}}
\def\satallax{\textsf{Satallax}}
\def\zenm{\textsf{Zenon~Modulo}}
\def\zenon{\textsf{Zenon}}
\def\zipper{\textsf{Zipperposition}}
\def\znot{\textsf{Z}}

\EnableBpAbbreviations{}
\newcommand{\UICm}[1]{\UIC{$#1$}}
\newcommand{\AXCm}[1]{\AXC{$#1$}}
\newcommand{\BICm}[1]{\BIC{$#1$}}
\newcommand{\TICm}[1]{\TIC{$#1$}}
\newcommand{\RLm}[1]{\RL{$#1$}}

\newcommand\clauseWithSubst[2]{\ensuremath{#1 ~|~ #2}}
\newcommand\renameVarsSymb{\ensuremath{\textsf{rename}}}
\newcommand\renameVars[1]{\ensuremath{\renameVarsSymb}(#1)}
\newcommand\mapVar[2]{\ensuremath{ #1 \mapsto #2 }}

\def\rew{\longrightarrow}
\def\arg{\mbox{-}}
\def\type{\textsf{Type}}
\def\omicron{o}
\newcommand\set[1]{\mathsf{set}(#1)}
\newcommand\tuple[2]{\mathsf{tup}(#1,#2)}
\def\plus{\raisebox{.22\height}{\scalebox{.6}{\pmb{+}}}}
\def\fctpart{~{\kern+.2em\mapstochar{}\!\!\!\!\!\rightarrow}~}
\def\injpart{~~{\mapstochar{}\!\!\!\!\!\!\rightarrowtail}~}
\def\surpart{~{\kern+.2em\mapstochar{}\!\!\!\!\!\twoheadrightarrow}~}
\def\bijpart{~{\kern+.1em\mapstochar{}\!\!\!\!\!\rightarrowtail\kern+.03em
\!\!\!\!\!\!\twoheadrightarrow}~}
\def\bijtotal{~{\kern-.1em\rightarrowtail\kern+.03em\!\!\!\!\!\!
\twoheadrightarrow}~}
\def\override{~~{\plus{}\mkern-24mu<}~}
\def\substleft{~{-\mkern-14mu\lhd}~}
\def\substright{~{-\mkern-14mu\rhd}~}

\newlength\memparindent
\setlength{\memparindent}{\parindent}
\newcommand{\myindent}{\hspace{\memparindent}}

\newcommand\todo[1]{\textcolor{red}{#1}}


\begin{document}

\title{SAT Solving Modulo Tableau Theory}
\titlerunning{SAT Solving Modulo Tableau Theory}

\author{Guillaume~Bury\inst{1} \and Simon~Cruanes\inst{2} \and
David~Delahaye\inst{3}}
\authorrunning{Guillaume~Bury, Simon~Cruanes, and David~Delahaye}
\tocauthor{Guillaume~Bury, Simon~Cruanes, and David~Delahaye}

\institute{\inria{}/\lsv{}/\ens{}, Cachan, France\\
\email{Guillaume.Bury@inria.fr} \and
\inria{}/\loria{}, Nancy, France\\
\email{Simon.Cruanes@inria.fr} \and
\lirmm{}/\um{}, Montpellier, France\\
\email{David.Delahaye@lirmm.fr}}

\maketitle

\begin{abstract}
We propose a new automated theorem prover that combines a SAT solver core with
tableau calculus. Tableau inference rules are used to unfold propositional
content into clauses while atomic formulas are treated using satisfiability
decision procedures as is done in SMT solvers. In order to treat quantified
first-order formulas, we use meta-variables and perform rigid unification modulo
equalities, for which we introduce an algorithm based on superposition, but
where all clauses contain a single atomic formula.

\keywords{Tableaux, SAT, SMT, Theorem Proving, Superposition}
\end{abstract}

% $Id$

\section{Introduction}

These days, Boolean satisfiability (SAT) solvers are undeniably quite reliable
tools despite the worst-case exponential run time of all known
algorithms. Modern SAT solvers provide a ``black-box'' procedure that can often
solve hard structured problems with over a million variables and several million
constraints. These tools are therefore used in many applications outside the
scope of knowledge representation and reasoning, which nonetheless was their
initial domain of application. The underlying representational formalism of SAT
solvers is propositional logic, and even if a given problem is not a pure
propositional problem, it is not unusual to encode it so that it can be viewed
as a propositional reasoning task. This encoding may be complex and possibly
costly, but the effectiveness of SAT solvers generally compensates for the cost
of this encoding.

This growing interest for SAT solvers has steered them toward a more first order
approach, with the possibility to deal with first order literals and reason
modulo theories. This resulted in the design of Satisfiability Modulo Theories
(SMT) solvers, which are built on top of SAT solvers. Over the last past few
years, SMT solvers have appeared as very efficient tools to reason over some
well identified theories (equality, uninterpreted functions, linear arithmetic,
arrays, etc.), and have allowed us to drag SAT solving toward first order logic.
Although modern SMT solvers supports first order logic, most of them use
heuristic quantifier instantiation for incorporating quantifier reasoning with
ground decision procedures. This mechanism is relatively effective in some cases
in practice, but it is not refutationally complete for first order logic. Hints
(triggers) are usually required, and it is sensitive to the syntactic structure
of the formula, so that it fails to prove formulas that can be easily discharged
by provers based on more traditional first order proof search methods (tableaux,
resolution, etc.).

In this paper, we propose to overcome this problem of completeness for first
logic in SMT solving by combining a SAT solver core with tableau calculus. The
tableau calculus is introduced as a regular theory of SMT solving, and the rules
are used to unfold propositional content into clauses while atomic formulas are
handled using satisfiability decision procedures as usually done in SMT solvers.
In order to manage quantified first order formulas, we use metavariables and
perform rigid unification modulo equalities, for which we introduce an algorithm
based on superposition, but where all clauses contain a single atomic formula.

We also propose to improve proof search by introducing rewriting to the SAT
solver as a regular SMT theory and along the lines of deduction modulo
theory. Deduction modulo theory~\cite{DA03} focuses on the computational part of
a theory, where axioms are transformed into rewrite rules, which induces a
congruence over propositions, and where reasoning is performed modulo this
congruence. In deduction modulo theory, this congruence is then induced by a set
of rewrite rules over both propositions and terms.

Our approach provides several advantages compared to usual SMT solving and first
order proof search methods. First, we benefit from the efficiency of a SAT
solver core together with a complete method of instantiation (when a
propositional model is found, we try to find a conflict between two literals by
unification). Second, it should be noted that our approach requires no change in
the architecture of the SMT solver, since the tableau calculus and rewriting are
seen as regular theories. Finally, no preliminary Skolemization and Conjunctive
Normal Form (CNF) transformation is required. This transformation is performed
lazily by applying the tableau rules progressively when a literal is propagated
or decided.  This makes the production of genuine output proofs easier, contrary
to the usual approach, where the Skolemization/CNF translation is realized at
the beginning and externalized with respect to the proof search.

Our proposal combining SAT solving with tableau calculus and rewriting has been
implemented and the corresponding tool is called \archsat{}. This tool is able
to deal with first order logic extended to polymorphic types à la ML, through a
type system in the spirit of~\cite{BP13}. To test this tool, we propose a
benchmark in the framework of the set theory of the \bmth{}
method~\cite{B-Book}. This theory~\cite{BA15} has been expressed using first
order logic extended to polymorphic types and turned into a theory that is
compatible with deduction modulo theory, i.e. where a large part of axioms has
been turned into rewrite rules. The benchmark itself gathers 319~lemmas coming
from Chap.~2 of the \bbook{}~\cite{B-Book}.

The paper is organized as follows: in Sec.~\ref{sec:smt}, we recall the classic
architecture of SMT solvers and we introduce the tableau and rewriting theories;
we then describe, in Sec.~\ref{sec:rusuper}, our mechanism of equational
reasoning by means of rigid unit superposition; finally, in
Sec.~\ref{sec:bench}, we present some experimental results obtained by running
our implementation over a benchmark of lemmas in the \bmth{} set theory, and
also provide related work in Sec.~\ref{sec:rwork}.

% $Id$

\section{SAT Solving Modulo Tableau Theory}

\todo{introduction, also explain why this is better than a
  regular (propositional) Tableau prover, and what the advantages are}
\todo{cite Satallax, brief comparison (w.r.t quantifiers)}

\subsection{SAT Solving Modulo Theories}

In this section, we recall the classic architecture of Satisfiability Modulo
Theory~\cite{barrett2006splitting} (SMT).
We introduce $\mathcal{T}$ and $\mathcal{F}$ respectively the sets of first
order terms and first order formulas over the signature
$\mathcal{S}=(\mathcal{S}_\mathcal{F},\mathcal{S}_\mathcal{P})$, where
$\mathcal{S}_\mathcal{F}$ is the set of function symbols, and
$\mathcal{S}_\mathcal{P}$, the set of predicate symbols, such that
$\mathcal{S}_\mathcal{F}\cap\mathcal{S}_\mathcal{P}=\emptyset$. The set
$\mathcal{T}$ is extended with two kinds of terms specific to tableau proof
search, i.e. $\epsilon{}$-terms (used instead of Skolemization) of the form
$\epsilon(x).P(x)$, where $P(x)$ is a formula, and which means some $x$ that
satisfies $P(x)$, if it exists, and metavariables (often named free variables in
the tableau-related literature) of the form $X_P$, where $P$ is the formula that
introduces the metavariable, and which is either $\forall{}x.Q(x)$ or
$\neg\exists{}x.Q(x)$, with $Q(x)$ a formula.

A boxed formula is of the form $\lfloor{}P\rfloor$, where $P$ is a formula. A
boxed formula is called atom, and a literal is either an atom, or the negation
of an atom. A literal is such that there is no negation on top of the boxed
formula (which means that $\lfloor\neg{}P\rfloor=\neg\lfloor{}P\rfloor$, and
that $\neg\neg\lfloor{}P\rfloor=\lfloor{}P\rfloor$). A clause is a disjunction
of literals. It should be noted that SAT solving usually reasons over sets of
clauses composed of first order literals; here, a literal is a first order
formula (possibly with quantifiers), which requires to box formulas to get a
regular SAT solving problem where boxed formulas are propositional variables.

An assignment of the boxed formulas is a partial function that assigns to boxed
formulas an element of $\{\top,\bot\}$ (values of the Boolean algebra). Given a
clause $C$, we say that $C$ is valid and write $\models{}C$, if $C$ evaluates to
$\top$ for every assignment. Given a set of clauses $S$, we say that $S$ is
valid and write $\models{}S$, if each clause of $S$ is valid. Given two sets of
clauses $S$ and $S'$, we say that $S$ implies $S'$ and write $S\models{}S'$ if,
for every assignment, if each clause of $S$ evaluates to $\top$ then each clause
of $S'$ evaluates to $\top$. If $S$ is a set of literals, then we say that $S$
is a model of $S'$.

A theory $T$ is a set of formulas over a given signature $\mathcal{S}$, which
are called axioms of the theory $T$. A clause $C$ is a tautology in the theory
$T$, written $\models_TC$, if the axioms of $T$ imply the formula $C'$, where
$C'$ is the clause $C$ where all the boxed formulas have been unboxed. A set of
clauses $S$ is a tautology in the theory $T$ if each clause of $S$ is a
tautology in $T$. Given two sets of clauses $M$ and $S$, we say that $M$ is a
model of $S$ modulo the theory $T$ and write $M\models_TS$, if $M$ is a model of
$S$ and $M$ is a tautology in $T$.

A SAT solver modulo theories is characterized by a set of theories $S_T$ and an
internal state of the form $M\parallel{}S$, where $M$ is an ordered list of
literals (if $M=l_1,l2,\ldots,l_n$ then $l_1<l_2<\ldots<l_n$) and $S$ a set of
clauses. Intuitively, $M$ represents the model of $S$, which will be built
progressively (initially $M=\emptyset$) by the application of a set of rules
over the internal state of the solver. This set of rules is presented in
Fig.~\ref{fig:smt}. Each time a literal is propagated (rule
``$\mathrm{unit~prop}$'') or decided (rule ``$\mathrm{decide}$''), it is given
to each theory, which may generate new clauses (rule ``$\mathrm{learn}$''). If a
clause is unvalidated by the model, then it is possible to backtrack the
decision over a given literal of the model if it exists (rule
``$\mathrm{backjump}$'').  There are two final states in this transition
system. The first one is ``$\mathrm{unsat}$'', which means that $S$ is
unsatisfiable (rule ``$\mathrm{unsat}$''). The second one is $M\parallel{}S$,
where $M$ is a model of $S$ and a tautology for every theory in $S_T$. In this
last case, if $M$ is not a tautology in a theory $T$, this theory generates a
set of conflict clauses (rule ``$\mathrm{learn}$'').

\begin{figure}[t]
\parbox{\textwidth}
{\small
\begin{center}
$\begin{array}{l@{\hspace{0.4cm}}l@{\hspace{0.4cm}}l}
M\parallel{}S,C\lor{}l\longrightarrow{}M,l\parallel{}S,C\lor{}l &
(\mathrm{unit~prop}) & \mbox{if }l\not\in{}M\mbox{ and }M\models{}\neg{}C\\\\

M\parallel{}S\longrightarrow{}M,l^\mathrm{d}\parallel{}S & (\mathrm{decide}) &
\mbox{if }l\not\in{}M\mbox{, and }l\in{}S\mbox{ or }\neg{}l\in{}S\\\\

M\parallel{}S,C\longrightarrow\mathrm{unsat} & (\mathrm{unsat}) &
\mbox{if }M'\models{}\neg{}C\mbox{ s.t. }M'\subseteq{}M\\
&& \mbox{and there is no }l^\mathrm{d}\leq{}l'\mbox{ in }M\\
&& \mbox{for all }l'\in{}M'\\\\
 
M,l^\mathrm{d},M'\parallel{}S,C\longrightarrow{}M,l'\parallel{}S,C &
(\mathrm{backjump}) & \mbox{if }M,l^\mathrm{d},M'\models\neg{}C
\mbox{, and there is}\\
&& \mbox{some clause }C'\lor{}l' s.t.:\\
&& l'\not\in{}M\mbox{, and }l'\in{}S\mbox{ or }\neg{}l'\in{}S\\
&& \mbox{or }l'\in{}M,l^\mathrm{d},M'\mbox{ or }
\neg{}l'\in{}M,l^\mathrm{d},M',\\
&& \mbox{and }S,C\models{}C'\lor{}l'\mbox{, and }M\models\neg{}C'\\\\

M\parallel{}S\longrightarrow{}M\parallel{}S,S'
& (\mathrm{learn}) & \models_TS'\mbox{, where }T\mbox{ is a theory}
\end{array}$
\end{center}}
\caption{Rules of SAT Solving Modulo Theory}
\label{fig:smt}
\end{figure}

\subsection{The Tableau Theory}

In the SAT solver previously described, the tableau proof search method is
integrated as a regular theory. When a literal is propagated or decided, we
generate a set of clauses corresponding to the application of a tableau rule
over the logical connective at root of the formula in the box of the
literal. More precisely, for a literal $l$, we generate the set of clauses
$\llbracket{}l\rrbracket$, where the function $\llbracket\cdot\rrbracket$ is
described by the rules of Fig.~\ref{fig:tabth}. When a literal is propagated or
decided, we use all the rules of Fig.~\ref{fig:tabth} except the instantiation
$\gamma$-rules (rules $\gamma_{\forall\mathrm{inst}}$ and
$\gamma_{\neg\exists\mathrm{inst}}$). It should be noted that we use the same
names for the rules than in tableau calculus ($\alpha$-rules, $\beta$-rules,
etc.), but there is no precedence between rules and therefore no priority in the
application of the rules contrary to the tableau proof search method (where
$\alpha$ rules are applied before $\beta$-rules, and so on).

When the SAT solver reaches a state $M\parallel{}S$, where $M$ is a model of
$S$, we look for a conflict in $M$ between two literals by unification and we
generate the clauses corresponding to the instantiation of the metavariables
using the result of the unification. More precisely, if there exist two literals
$l$ and $\neg{}l'$ in $M$ such that $l=\lfloor{}Q\rfloor$ and
$l'=\lfloor{}R\rfloor$, with $Q$ and $R$ two formulas, then for each
substitution $(X_{\forall{}x.P(x)}\mapsto{}t)\in\mathrm{mgu}(Q,R)$ (resp.
$(X_{\neg\exists{}x.P(x)}\mapsto{}t)\in\mathrm{mgu}(Q,R)$) such that there is no
$Y_T\in\forall{}x.P(x)$ (resp. $Y_T\in\neg\exists{}x.P(x)$), we can generate the
clauses $\llbracket\lfloor\forall{}x.P(x)\rfloor{}\rrbracket$ 
(resp. $\llbracket\neg\lfloor\exists{}x.P(x)\rfloor{}\rrbracket$) using the rule
$\gamma_{\forall\mathrm{inst}}$ (resp. $\gamma_{\neg\exists\mathrm{inst}}$) of
Fig.~\ref{fig:tabth}.

\begin{figure}[t]
\parbox{\textwidth}
{\small
\underline{Analytic Rules}
\begin{center}
$\begin{array}{lll@{\hspace{0.5cm}}l}
\llbracket\lfloor{}P\land{}Q\rfloor{}\rrbracket & = &
\neg\lfloor{}P\land{}Q\rfloor\lor\lfloor{}P\rfloor,
\neg\lfloor{}P\land{}Q\rfloor\lor\lfloor{}Q\rfloor & (\alpha_\land)\\\\

\llbracket\neg\lfloor{}P\land{}Q\rfloor{}\rrbracket & = &
\lfloor{}P\land{}Q\rfloor\lor\neg\lfloor{}P\rfloor\lor\neg\lfloor{}Q\rfloor &
(\beta_{\neg\land})\\\\

\llbracket{}\lfloor{}P\lor{}Q\rfloor{}\rrbracket & = &
\neg\lfloor{}P\lor{}Q\rfloor\lor\lfloor{}P\rfloor\lor\lfloor{}Q\rfloor &
(\beta_\lor)\\\\

\llbracket\neg\lfloor{}P\lor{}Q\rfloor{}\rrbracket & = &
\lfloor{}P\lor{}Q\rfloor\lor\neg\lfloor{}P\rfloor,
\lfloor{}P\lor{}Q\rfloor\lor\neg\lfloor{}Q\rfloor &
(\alpha_{\neg\lor})\\\\

\llbracket{}\lfloor{}P\Rightarrow{}Q\rfloor{}\rrbracket & = &
\neg\lfloor{}P\Rightarrow{}Q\rfloor\lor\neg\lfloor{}P\rfloor\lor
\lfloor{}Q\rfloor & (\beta_\Rightarrow)\\\\

\llbracket\neg\lfloor{}P\Rightarrow{}Q\rfloor{}\rrbracket & = &
\lfloor{}P\Rightarrow{}Q\rfloor\lor\lfloor{}P\rfloor,
\lfloor{}P\Rightarrow{}Q\rfloor\lor\neg\lfloor{}Q\rfloor &
(\alpha_{\neg\Rightarrow})\\\\

\llbracket{}\lfloor{}P\Leftrightarrow{}Q\rfloor{}\rrbracket & = &
\neg\lfloor{}P\Leftrightarrow{}Q\rfloor\lor\lfloor{}P\Rightarrow{}Q\rfloor,
\neg\lfloor{}P\Leftrightarrow{}Q\rfloor\lor\lfloor{}Q\Rightarrow{}P\rfloor &
(\beta_\Rightarrow)\\\\

\llbracket{}\neg\lfloor{}P\Leftrightarrow{}Q\rfloor{}\rrbracket &
= &
\lfloor{}P\Leftrightarrow{}Q\rfloor\lor\neg\lfloor{}P\Rightarrow{}Q\rfloor\lor
\neg\lfloor{}Q\Rightarrow{}P\rfloor & (\beta_{\neg\Rightarrow})
\end{array}$
\end{center}

\underline{$\delta$-Rules}
\begin{center}
$\begin{array}{lll@{\hspace{0.5cm}}l}
\llbracket\lfloor\exists{}x.P(x)\rfloor{}\rrbracket & = &
\neg{}\lfloor\exists{}x.P(x)\rfloor\lor\lfloor{}P(\epsilon(x).P(x))\rfloor &
(\delta_\exists)\\\\

\llbracket\neg\lfloor\forall{}x.P(x)\rfloor{}\rrbracket & = &
\lfloor\forall{}x.P(x)\rfloor\lor\neg\lfloor{}P(\epsilon(x).\neg{}P(x))\rfloor &
(\delta_{\neg\forall})
\end{array}$
\end{center}

\underline{$\gamma$-Rules}
\begin{center}
$\begin{array}{lll@{\hspace{0.5cm}}l}
\llbracket\lfloor\forall{}x.P(x)\rfloor{}\rrbracket & = &
\neg\lfloor\forall{}x.P(x)\rfloor\lor\lfloor{}P(X_{\forall{}x.P(x)})\rfloor &
(\gamma_{\forall{}M})\\\\

\llbracket\neg\lfloor\exists{}x.P(x)\rfloor{}\rrbracket & = &
\lfloor\exists{}x.P(x)\rfloor\lor
\neg\lfloor{}P(X_{\neg\exists{}x.P(x)})\rfloor &
(\gamma_{\neg\exists{}M})\\\\

\llbracket\lfloor\forall{}x.P(x)\rfloor{}\rrbracket & = &
\neg\lfloor\forall{}x.P(x)\rfloor\lor\lfloor{}P(t)\rfloor &
(\gamma_{\forall\mathrm{inst}})\\\\

\llbracket\neg\lfloor\exists{}x.P(x)\rfloor{}\rrbracket & = &
\lfloor\exists{}x.P(x)\rfloor\lor\neg\lfloor{}P(t)\rfloor &
(\gamma_{\neg\exists\mathrm{inst}})
\end{array}$
\end{center}}
\caption{Rules of Tableau Theory}
\label{fig:tabth}
\end{figure}

To show how this theory works, let us prove that
$\exists{}x.P(x)\Rightarrow{}P(a)\land{}P(b)$, where $P$ is a predicate symbol
and $a,b$ two constants. The SAT solver is initiated with the state
$\emptyset\parallel
\neg\lfloor\exists{}x.P(x)\Rightarrow{}P(a)\land{}P(b)\rfloor$, and the proof is
described in Fig.~\ref{fig:exa}, where $X$ is a shortcut for
$X_{\neg\exists{}x.P(x)\Rightarrow{}P(a)\land{}P(b)}$. It should be noted that
to do this proof in sequent calculus, a right contraction is necessary to
instantiate the formula twice (with $a$ and $b$). In Fig.~\ref{fig:exa}, it is
done by propagating the literal $\lfloor{}P(X)\rfloor{}$, which can provide as
many instantiations (by unification) as necessary.

\begin{figure}[t!]
\parbox{\textwidth}
{\small
\begin{center}
$\begin{array}{lcl}
\emptyset\parallel
\neg(A\equiv\lfloor\exists{}x.P(x)\Rightarrow{}P(a)\land{}P(b)\rfloor) &
\longrightarrow & (\mathrm{unit~prop})\\

\neg{}A\parallel{}\neg{}A & \longrightarrow & (\mathrm{learn})\\

\neg{}A\parallel
\neg{}A,A\lor\neg(B\equiv\lfloor{}P(X)\Rightarrow{}P(a)\land{}P(b)\rfloor) &
\longrightarrow & (\mathrm{unit~prop})\\

\neg{}A,\neg{}B\parallel\neg{}A,A\lor\neg{}B & \longrightarrow &
(\mathrm{learn})\\

\neg{}A,\neg{}B\parallel\neg{}A,A\lor\neg{}B,
B\lor{}(C\equiv\lfloor{}P(X)\rfloor{}),\\
~~~~B\lor\neg(D\equiv\lfloor{}P(a)\land{}P(b)\rfloor{}) &
\longrightarrow & (\mathrm{unit~prop})\times{}2\\

\neg{}A,\neg{}B,C,\neg{}D\parallel\neg{}A,A\lor\neg{}B,B\lor{}C,B\lor\neg{}D &
\longrightarrow & (\mathrm{learn})\\

\neg{}A,\neg{}B,C,\neg{}D\parallel\neg{}A,A\lor\neg{}B,B\lor{}C,B\lor\neg{}D,\\
~~~~D\lor\neg(E\equiv\lfloor{}P(a)\rfloor)\lor
\neg(F\equiv\lfloor{}P(b)\rfloor) & \longrightarrow & (\mathrm{decide})\\

\neg{}A,\neg{}B,C,\neg{}D,\neg{}E^d\parallel
\neg{}A,A\lor\neg{}B,B\lor{}C, & \longrightarrow &
(\mathrm{learn})\\
~~~~B\lor\neg{}D,D\lor\neg{}E\lor\neg{}F && \{X\mapsto{}a\}=\mathrm{mgu}(C,E)\\

\neg{}A,\neg{}B,C,\neg{}D,\neg{}E^d\parallel
\neg{}A,A\lor\neg{}B,B\lor{}C,\\
~~~~B\lor\neg{}D,D\lor\neg{}E\lor\neg{}F,\\
~~~~A\lor\neg(G\equiv{}\lfloor{}P(a)\Rightarrow{}P(a)\land{}P(b)\rfloor) &
\longrightarrow & (\mathrm{unit~prop})\\

\neg{}A,\neg{}B,C,\neg{}D,\neg{}E^d,\neg{}G\parallel
\neg{}A,A\lor\neg{}B,B\lor{}C,\\
~~~~B\lor\neg{}D,D\lor\neg{}E\lor\neg{}F,A\lor\neg{}G & \longrightarrow &
(\mathrm{learn})\\

\neg{}A,\neg{}B,C,\neg{}D,\neg{}E^d,\neg{}G\parallel
\neg{}A,A\lor\neg{}B,B\lor{}C,\\
~~~~B\lor\neg{}D,D\lor\neg{}E\lor\neg{}F,A\lor\neg{}G,G\lor{}E,G\lor\neg{}D &
\longrightarrow & (\mathrm{backjump})\\

\neg{}A,\neg{}B,C,\neg{}D,E\parallel
\neg{}A,A\lor\neg{}B,B\lor{}C,\\
~~~~B\lor\neg{}D,D\lor\neg{}E\lor\neg{}F,A\lor\neg{}G,G\lor{}E,G\lor\neg{}D &
\longrightarrow & (\mathrm{unit~prop})\times{}2\\

\neg{}A,\neg{}B,C,\neg{}D,E,\neg{}F,\neg{}G\parallel
\neg{}A,A\lor\neg{}B,B\lor{}C, & \longrightarrow & (\mathrm{learn})\\
~~~~B\lor\neg{}D,D\lor\neg{}E\lor\neg{}F,A\lor\neg{}G,G\lor{}E,G\lor\neg{}D &&
\{X\mapsto{}b\}=\mathrm{mgu}(C,F)\\

\neg{}A,\neg{}B,C,\neg{}D,E,\neg{}F,\neg{}G\parallel
\neg{}A,A\lor\neg{}B,B\lor{}C,\\
~~~~B\lor\neg{}D,D\lor\neg{}E\lor\neg{}F,A\lor\neg{}G,G\lor{}E,G\lor\neg{}D\\
~~~~A\lor\neg(H\equiv{}\lfloor{}P(b)\Rightarrow{}P(a)\land{}P(b)\rfloor) &
\longrightarrow & (\mathrm{unit~prop})\\

\neg{}A,\neg{}B,C,\neg{}D,E,\neg{}F,\neg{}G,\neg{}H\parallel
\neg{}A,A\lor\neg{}B,B\lor{}C,\\
~~~~B\lor\neg{}D,D\lor\neg{}E\lor\neg{}F,A\lor\neg{}G,G\lor{}E,G\lor\neg{}D\\
~~~~A\lor\neg{}H & \longrightarrow & (\mathrm{learn})\\

\neg{}A,\neg{}B,C,\neg{}D,E,\neg{}F,\neg{}G,\neg{}H\parallel
\neg{}A,A\lor\neg{}B,B\lor{}C,\\
~~~~B\lor\neg{}D,D\lor\neg{}E\lor\neg{}F,A\lor\neg{}G,G\lor{}E,G\lor\neg{}D\\
~~~~A\lor\neg{}H,H\lor{}F,H\lor{}\neg{}D & \longrightarrow &
(\mathrm{unsat})\\

\mathrm{unsat}
\end{array}$
\end{center}}
\caption{Example of Proof}
\label{fig:exa}
\end{figure}

% $Id$

\section{Equational Reasoning with Rigid Unit Superposition}

\EnableBpAbbreviations{}

\newcommand\clauseWithSubst[2]{\ensuremath{#1 ~|~ #2}}

There are many ways of integrating equational reasoning in
Tableaux methods~\cite{brand1975proving,letz2002integration,backeman2015theorem,degtyarev1996you}.
Because our prover does not rely on clausal forms, but on arbitrary formulas
with quantifiers occurring deep inside branches,
we deal with {\em rigid} variables --- variables that can be instantiated
only once.
The problem we want to solve, {\em rigid E-unification}, is the following.
Assume a set of equations $E$, containing rigid variables,
and a {\em target equation} $e = s \approx t$.
We want a substitution $\sigma$ such that
$\bigwedge_{e \in E} e\sigma \vdash_\approx s\sigma \approx t\sigma$;
such a substitution is a {\em solution} to the rigid E-unification problem.

We propose here an approach based on superposition with rigid variables,
as in previous work by Degtyarev and Voronkov~\cite{degtyarev1996you}
and earlier work on rigid paramodulation~\cite{plaisted1995special},
but with significant differences.
First, to avoid constraint solving, we do not use
basic superposition nor constraints.
Second, we introduce a {\em merging} rule that factors together
intermediate (dis)equations that are alpha-equivalent:
with multiple instances of some of the quantified formulas ({\em amplification}),
it becomes important not to duplicate work.
Third, unlike rigid paramodulation we use a term ordering to orient the
equations.

\subsection{Preliminary Definitions}
%{{{


We write $ \clauseWithSubst{ s \approx t }{ \Sigma}$
(resp. $ \clauseWithSubst{ s \not\approx t }{ \Sigma}$),
the unit clause that contains exactly one equation (resp.~disequation)
under hypothesis $\Sigma$ (which is a set of substitutions).
We write $\clauseWithSubst{\emptyset}{\Sigma}$ for the empty clause under hypothesis $\Sigma$.
The E-unification problem $E \vdash s\approx t$ can be solved by
proving $\clauseWithSubst{\emptyset }{ \Sigma}$
from $\{ \clauseWithSubst{e }{ \{ \emptyset \} } \}_{ e \in E }
\cup
\{ \clauseWithSubst{s \not\approx t }{ \{ \emptyset \} } \}$,
where $\Sigma$ contains the solutions.
The meaning of $s \approx t | \Sigma$ is that for every $\sigma \in \Sigma$,
$s\sigma \approx t\sigma$ is provable.
We keep a set of substitutions, rather than unit clauses paired with
individual substitutions, in order to avoid duplicating the work
for alpha-equivalent clauses.
Indeed, because of amplification, many instances of a given (dis)equation
might be present in a branch of the tableau;
it would be inefficient to repeat the same inference steps with each variant
of the axioms.

To perform inferences between two unit (dis)equations, we need to
merge their sets of substitutions.
Therefore, we need a notion of compatibility on substitutions, using a
partial ordering $\leq$, such that $\sigma \leq \sigma'$ means that $\sigma$ is
less general than $\sigma'$.
We start by defining a few notions.

Considering a substitution as a function from variables to terms, we can define
the domain of a substitution $\sigma$ as the set of variables which have a non-trivial
binding in $\sigma$.\footnote{a trivial binding maps a variable to itself.}
The co-domain of a substitution is the set of variables occurring in terms in the image of
the domain of the substitution.
In all the following, we will consider idempotent substitutions, i.e.~substitution for which
the domain and co-domain have an empty intersection.

The {\em composition} of substitutions $\sigma \circ \sigma'$
is well-defined iff the domains of $\sigma$ and $\sigma'$ have no
intersection.
In this case,
$\sigma \circ \sigma' \triangleq \left\{ x \mapsto \sigma'(\sigma(x)) | x \in \text{domain}(\sigma) \right\}$
This definition naturally extends to sets of substitutions:
$\Sigma \circ \sigma' \triangleq \left\{ \sigma \circ \sigma' | \sigma \in \Sigma \right\}$
We then say that $\sigma \leq \sigma'$ iff $\exists \sigma''.~ \sigma \circ \sigma'' = \sigma'$.
We extend that notion to set of substitutions:
$\smash{ \Sigma \leq \Sigma' }$
iff $\smash{ \forall \sigma' \in \Sigma'.~ \exists \sigma \in \Sigma. \sigma \leq \sigma' }$.
The {\em merging} of two substitutions $\sigma \uparrow \sigma'$ as the supremum of $\{\sigma,\sigma'\}$
for the order $\leq$, if it exists, or $\bot$ otherwise.
The merging of sets of substitutions is
$\Sigma \uparrow \Sigma' \triangleq
  \left\{ \sigma \uparrow \sigma' ~|~
    \sigma \in \Sigma, \sigma' \in \Sigma' \right.,
  \sigma \uparrow \sigma' \not= \bot
  \}$.
An inference rule only succeeds if the merging of the premises' substitution
sets is non-empty.

%}}}

\subsection{Inference System}
%{{{

In Figure~\ref{fig:unit-sup-rules}, we present the  rules for unit superposition
with rigid variables.
We adapt notations and names from Schulz's paper on E~\cite{e_brainiac_prover}.
A single bar denotes an inference --- we add the result to the saturation set ---
whereas a double bar is a simplification in which the premises are
replaced by the conclusion(s).
The relation $\prec$ is a {\em reduction ordering}, used to orient equations,
thus pruning the search space.
Typically, $\prec$ is one of RPO or KBO.

\begin{description}
%{{{
  \item[ER] is {\em equality resolution},
    where a disequation $\clauseWithSubst{s \not\approx t}\Sigma$
    is solved by syntactically unifying $s$ and $t$ with $\sigma$,
    if $\sigma$ is compatible with $\Sigma$.
  \item[SN] is superposition into negative literals. A subterm of $u$
    is rewritten using $s \approx t$ after unifying it with $s$
    by $\sigma$.
    The rewriting is done only if $s\sigma \not\preceq t\sigma$,
    a sufficient (but not necessary) condition for a
    ground instance of $s\sigma \approx t\sigma$
    to be oriented left-to-right.
  \item[SP] is similar to SN, but superposes into a positive literal.
  \item[TD1] deletes trivial equations that will never contribute to a proof.
  \item[TD2] deletes clauses with an empty set of substitutions.
    In practice, we only apply a rule if the conclusion is labelled with a
    non-empty set of substitutions.
  \item[ME] merges two alpha-equivalent clauses into a single clause,
    by merging the sets of substitutions.
    This rule is very important in practice, to prevent the search space
    from exploding due to the duplicates of most formulas.
    Superposition deals with this explosion by removing duplicates using
    {\em subsumption}, but in our context subsumption is not complete
    because rigid variables are only proxy for ground terms:
    even if $C\sigma \subseteq D$, the one ground instance of $C$ might not
    be compatible with the ground instance of $D$.
  \item[ES] is a restricted form of equality subsumption. The active
    equation $\clauseWithSubst{ s\approx t}\Sigma $ can be used to delete another clause, as
    in E~\cite{e_brainiac_prover}.
    However, ES only works if $s$ and $t$ are syntactically equal to the
    corresponding subterms in the subsumed clause $C$; otherwise, there is no
    guarantee that further instantiations will not make
    $s\approx t$ incompatible with $C$.
    Moreover, $C$ needs not be entirely removed; only its substitutions
    that are compatible with $\Sigma$ are actually subsumed.
  \item[RP] similarly, rewriting of positive clauses only  works for
    syntactical equality, not matching.
  \item[RN] is the same as RP but for rewriting negative clauses.
%}}}
\end{description}

\begin{figure}[htb]
%{{{
  \begin{center}

    % ER
    \AXC{$s \not\approx t |\Sigma$}
    \LL{ER}
    \RL{if $\sigma = \text{mgu}(s, t)$}
    \UIC{$\emptyset | \Sigma \circ \sigma $}
    \DP{} \\[12pt]

    % SN
    \AXC{$s \approx t | \Sigma$}
    \AXC{$u \not\approx v | \Sigma'$}
    \LL{SN}
    \BIC{$\sigma''(u[p \leftarrow t] \not\approx v) | (\Sigma \circ \sigma'') \uparrow (\Sigma' \circ \sigma'')$}
    \DP{}
    $\text{if} \left\{ \begin{array}{l}
        \sigma'' = \text{mgu}(u_{|p}, s) \\
        \sigma''(s) \not\preceq \sigma''(t) \\
        \sigma''(u) \not\preceq \sigma''(v) \\
        u_{|p} \not\in V \\
    \end{array}\right.$ \\[12pt]

    % SP
    \AXC{$s \approx t | \Sigma$}
    \AXC{$u \approx v | \Sigma'$}
    \LL{SP}
    \BIC{$\sigma''(u[p \leftarrow t] \approx v) | (\Sigma \circ \sigma'') \uparrow (\Sigma' \circ \sigma'')$}
    \DP{}
    $\text{if} \left\{ \begin{array}{l}
        \sigma'' = \text{mgu}(u_{|p}, s) \\
        \sigma''(s) \not\preceq \sigma''(t) \\
        \sigma''(u) \not\preceq \sigma''(v) \\
        u_{|p} \not\in V \\
    \end{array}\right.$ \\[12pt]

    \mbox{

    % TD1
    \AXC{$s \approx s | \Sigma $}
    \LL{TD1}
    \doubleLine{}
    \UIC{$\top$}
    \DP{}

    % TD2
    \AXC{$s \mathrel{R} t | \emptyset$}
    \LL{TD2}
    \RL{$ R \in \{ \approx, \not\approx \} $}
    \doubleLine{}
    \UIC{$\top$}
    \DP{}
  }
  \\[12pt]

    % PS
    %\AXC{$s \approx t |\sigma$}
    %\AXC{$u[p \leftarrow \sigma''(s)] \not\approx u[p \leftarrow \sigma''(t)] | \sigma' $}
    %\LL{PS}
    %\doubleLine{}
    %\BIC{$s \approx t | \sigma$ \qquad $\emptyset | \sigma \cup \sigma' \cup \sigma''$}
    %\DP{} \\[12pt]

    % NS
    %\AXC{$s \not\approx t | \sigma$}
    %\AXC{$\sigma''(s \approx t) | \sigma'$}
    %\LL{NS}
    %\doubleLine{}
    %\BIC{$s \not\approx t | \sigma$ \qquad $\emptyset | \sigma \cup \sigma' \cup \sigma''$}
    %\DP{} \\[12pt]

    % ME
    \AXC{$\rho(u) \approx \rho(v) | \Sigma$}
    \AXC{$u \approx v | \Sigma'$}
    \LL{ME}
    \RL{$\rho \text{ is a variable renaming}$}
    \doubleLine{}
    \BIC{$\rho(u) \approx \rho(v) | \Sigma \cup (\Sigma' \circ \rho)$}
    \DP{} \\[12pt]

    % ES
    \AXC{$s \approx t | \Sigma$}
    \AXC{$u[p \leftarrow s] \approx u[p \leftarrow t] | \Sigma' \cup \Sigma''$}
    \LL{ES}
    \RL{$
      \text{if} \left\{ \begin{array}{l}
          \Sigma'' \not= \emptyset \\
          \Sigma \leq \Sigma''
        \end{array}\right.  $}
    \doubleLine{}
    \BIC{$s\approx t | \Sigma \qquad u[p\leftarrow s] \approx u[p \leftarrow t] | \Sigma'$}
    \DP{} \\[12pt]

    % RP
    \AXC{$s \approx t | \Sigma$}
    \AXC{$u \approx v | \Sigma'$}
    \LL{RP}
    \doubleLine{}
    \BIC{$s \approx t | \Sigma$ \qquad $u[p \leftarrow t] \approx v | \Sigma'$}
    \DP{}
    $\text{if} \left\{\begin{array}{l}
      u_{|p} = s \\
      s \succ t \\
      \Sigma \leq \Sigma'\\
      u \not\succeq v ~ \text{or} ~ p \neq \lambda \\
    \end{array}\right.$ \\[12pt]

    % RN
    \AXC{$s \approx t | \emptyset$}
    \AXC{$u \not\approx v | \sigma$}
    \LL{RN}
    \doubleLine{}
    \BIC{$s \approx t | \emptyset$ \qquad $u[p \leftarrow t] \not\approx | \sigma$}
    \DP{}
    $\text{if} \left\{\begin{array}{l}
      u_{|p} = s \\
      s \succ t \\
      \Sigma \leq \Sigma'\\
    \end{array}\right.$

  \caption{The set of rules for unit rigid superposition}
  \label{fig:unit-sup-rules}
  \end{center}
%}}}
\end{figure}

%}}}

\subsection{Main Loop}
%{{{

Our objective with rigid E-unification is to attempt to close a branch
of the tableau prover (i.e., a set of boolean literals set to true).
To do so, all equational or atomic literals are added to a set of unit clauses
to process, with a label $\Sigma \triangleq \{ \emptyset \}$.
Then, the given-clause algorithm is applied to try and saturate the set.
Assuming a fair strategy, this will eventually find a solution
(i.e. derive $\clauseWithSubst{\emptyset }{\Sigma}$) if there exists one.
We refer the interested reader to~\cite{e_brainiac_prover} for more details.

Because the whole branch is treated by a single given-clause saturation loop,
we look for all solutions susceptible to close the branch at the same time.
Moreover, this technique is amenable to incrementality --- every time a
(dis)equation is decided by the SAT solver, we could add it to the saturation
set and perform a (limited) number of steps of the given-clause algorithm.

%}}}


\bibliographystyle{abbrv}
\bibliography{biblio}

\end{document}
