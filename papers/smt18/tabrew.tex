% $Id$

\section{SAT Solving Modulo Tableau and Rewriting Theories}
\label{sec:smt}

Compared to genuine tableau automated theorem provers, like \princess{} or
\zenon{} for example, our approach has the benefit of being versatile since the
tableau rules are actually integrated as a regular SMT theory. This way, the
tableau rules can be easily combined with other theories, such as equality logic
with uninterpreted functions or arithmetic. The way we integrate the tableau
rules into the SAT solver (mainly by boxing/unboxing first order formulas) is
close to what is done in the \satallax{} tool~\cite{CEB12}. The difference
resides in the fact that we are in a pure first order framework, which has
significant consequences in the management of quantifiers and unification in
particular (see Sec.~\ref{sec:super}).

Regarding the integration of rewriting, automated theorem provers rely on
several solutions (superposition rule for first order provers, triggers for SMT
solvers, etc.). But deduction modulo theory~\cite{DA03} is probably the most
general approach, where a theory can be partly turned into a set of rewrite
rules over both terms and propositions. Several proof search methods have been
extended to deduction modulo theory, resulting in tools such as \iproverm{} and
\zenm{}. This paper can be seen as a continuation of these previous experiments
adapted to the framework of SMT solving.

\subsection{The Tableau Theory}
\label{sec:tab}

We introduce $\mathcal{T}$ and $\mathcal{F}$ respectively,
the sets of first order terms and formulas over the signature
$\mathcal{S}=(\mathcal{S}_\mathcal{F},\mathcal{S}_\mathcal{P})$, where
$\mathcal{S}_\mathcal{F}$ is the set of function symbols, and
$\mathcal{S}_\mathcal{P}$, the set of predicate symbols, such that
$\mathcal{S}_\mathcal{F}\cap\mathcal{S}_\mathcal{P}=\emptyset$. The set
$\mathcal{T}$ is extended with two kinds of terms specific to tableau proof
search. First are $\epsilon{}$-terms (used instead of Skolemization) of the form
$\epsilon(x).P(x)$, where $P(x)$ is a formula, and which means some $x$ that
satisfies $P(x)$, if it exists. And second, metavariables (often named free variables in
the tableau-related literature) of the form $X_P$, where $P$ is the formula that
introduces the metavariable, and which is either $\forall{}x.Q(x)$ or
$\neg\exists{}x.Q(x)$, with $Q(x)$ a formula.

A boxed formula is of the form $\lfloor{}P\rfloor$, where $P$ is a formula. A
boxed formula is called an atom, and a literal is either an atom, or the negation
of an atom. A literal is such that there is no negation on top of the boxed
formula (which means that $\lfloor\neg{}P\rfloor=\neg\lfloor{}P\rfloor$, and
$\neg\neg\lfloor{}P\rfloor=\lfloor{}P\rfloor$). A clause is a disjunction of
literals. It should be noted that SMT solving usually reasons over sets of
clauses composed of first order literals; here, a literal is a first order
formula (possibly with quantifiers), which requires to box formulas to get a
regular SAT solving problem where boxed formulas are propositional variables.

Tableau proof search method is integrated as a regular theory in our SMT solver.
When a literal is propagated or decided, we
generate a set of clauses corresponding to the application of a tableau rule
depending on the logical connective at the root of the formula in the box of the
literal. More precisely, for a literal $l$, we generate the set of clauses
$\llbracket{}l\rrbracket$, where the function $\llbracket\cdot\rrbracket$ is
described by the rules of Fig.~\ref{fig:tabth}. When a literal is propagated or
decided, we use all the rules of Fig.~\ref{fig:tabth} except the instantiation
$\gamma$-rules (rules $\gamma_{\forall\mathrm{inst}}$ and
$\gamma_{\neg\exists\mathrm{inst}}$). It should be noted that we use the same
names for the rules as in tableau calculus ($\alpha$-rules, $\beta$-rules,
etc.), but there is no precedence between rules and therefore no priority in the
application of the rules contrary to the tableau proof search method (where
$\alpha$ rules are applied before $\beta$-rules, and so on).

When the SMT solver finds a model $M$ of the current set of clauses,
we look for a conflict in $M$ between boxed atomic formulas by unification and we
generate the clauses corresponding to the instantiation of the metavariables
using the result of the unification. More precisely, if there exist two literals
$l$ and $\neg{}l'$ in $M$ such that $l=\lfloor{}Q\rfloor$ and
$l'=\lfloor{}R\rfloor$, with $Q$ and $R$ two formulas, then for each binding
$(X_{\forall{}x.P(x)}\mapsto{}t)\in\sigma$ belonging to a
substitution\footnote{In practice, the implementation also splits and
reduce substitutions in order to only generate clauses corresponding to the
outermost meta-variables.}
$\sigma\in\mathrm{mgu}(Q,R)$ (resp.
$(X_{\neg\exists{}x.P(x)}\mapsto{}t)\in\sigma\in\mathrm{mgu}(Q,R)$)
we can generate the clauses $\llbracket\lfloor\forall{}x.P(x)\rfloor{}\rrbracket$
(resp. $\llbracket\neg\lfloor\exists{}x.P(x)\rfloor{}\rrbracket$) using the rule
$\gamma_{\forall\mathrm{inst}}$ (resp. $\gamma_{\neg\exists\mathrm{inst}}$) of
Fig.~\ref{fig:tabth}.

\begin{figure}[t]
\parbox{\textwidth}
{\small
\underline{Analytic Rules}
\begin{center}
$\begin{array}{llll}
(\alpha_\land) &
\llbracket\lfloor{}P\land{}Q\rfloor{}\rrbracket =
\left\{\begin{array}{l}
  \neg\lfloor{}P\land{}Q\rfloor\lor\lfloor{}P\rfloor \\
  \neg\lfloor{}P\land{}Q\rfloor\lor\lfloor{}Q\rfloor
\end{array} &

(\beta_{\neg\land}) &
\llbracket\neg\lfloor{}P\land{}Q\rfloor{}\rrbracket =
\lfloor{}P\land{}Q\rfloor\lor\neg\lfloor{}P\rfloor\lor\neg\lfloor{}Q\rfloor

\\\\

(\beta_\lor) &
\llbracket{}\lfloor{}P\lor{}Q\rfloor{}\rrbracket =
\neg\lfloor{}P\lor{}Q\rfloor\lor\lfloor{}P\rfloor\lor\lfloor{}Q\rfloor &

(\alpha_{\neg\lor}) &
\llbracket\neg\lfloor{}P\lor{}Q\rfloor{}\rrbracket =
\left\{\begin{array}{l}
  \lfloor{}P\lor{}Q\rfloor\lor\neg\lfloor{}P\rfloor \\
  \lfloor{}P\lor{}Q\rfloor\lor\neg\lfloor{}Q\rfloor
\end{array}

\\\\

(\beta_\Rightarrow) &
\llbracket{}\lfloor{}P\Rightarrow{}Q\rfloor{}\rrbracket =
\neg\lfloor{}P\Rightarrow{}Q\rfloor\lor\neg\lfloor{}P\rfloor\lor
\lfloor{}Q\rfloor &

(\alpha_{\neg\Rightarrow}) &
\llbracket\neg\lfloor{}P\Rightarrow{}Q\rfloor{}\rrbracket =
\left\{\begin{array}{l}
  \lfloor{}P\Rightarrow{}Q\rfloor\lor\lfloor{}P\rfloor \\
  \lfloor{}P\Rightarrow{}Q\rfloor\lor\neg\lfloor{}Q\rfloor
\end{array}

\\\\

(\beta_\Rightarrow) &
\llbracket{}\lfloor{}P\Leftrightarrow{}Q\rfloor{}\rrbracket =
\left\{\begin{array}{l}
  \neg\lfloor{}P\Leftrightarrow{}Q\rfloor\lor\lfloor{}P\Rightarrow{}Q\rfloor \\
  \neg\lfloor{}P\Leftrightarrow{}Q\rfloor\lor\lfloor{}Q\Rightarrow{}P\rfloor
\end{array} &

(\beta_{\neg\Rightarrow}) &
\llbracket{}\neg\lfloor{}P\Leftrightarrow{}Q\rfloor{}\rrbracket =
\begin{array}{l}
  \lfloor{}P\Leftrightarrow{}Q\rfloor \\
  \lor\neg\lfloor{}P\Rightarrow{}Q\rfloor \\
  \lor\neg\lfloor{}Q\Rightarrow{}P\rfloor
\end{array}
\end{array}$
\end{center}

\underline{$\delta$-Rules}
\begin{center}
$\begin{array}{lll@{\hspace{0.5cm}}l}
\llbracket\lfloor\exists{}x.P(x)\rfloor{}\rrbracket & = &
\neg{}\lfloor\exists{}x.P(x)\rfloor\lor\lfloor{}P(\epsilon(x).P(x))\rfloor &
(\delta_\exists)\\\\

\llbracket\neg\lfloor\forall{}x.P(x)\rfloor{}\rrbracket & = &
\lfloor\forall{}x.P(x)\rfloor\lor\neg\lfloor{}P(\epsilon(x).\neg{}P(x))\rfloor &
(\delta_{\neg\forall})
\end{array}$
\end{center}

\underline{$\gamma$-Rules}
\begin{center}
$\begin{array}{lll@{\hspace{0.5cm}}l}
\llbracket\lfloor\forall{}x.P(x)\rfloor{}\rrbracket & = &
\neg\lfloor\forall{}x.P(x)\rfloor\lor\lfloor{}P(X_{\forall{}x.P(x)})\rfloor &
(\gamma_{\forall{}M})\\\\

\llbracket\neg\lfloor\exists{}x.P(x)\rfloor{}\rrbracket & = &
\lfloor\exists{}x.P(x)\rfloor\lor
\neg\lfloor{}P(X_{\neg\exists{}x.P(x)})\rfloor &
(\gamma_{\neg\exists{}M})\\\\

\llbracket\lfloor\forall{}x.P(x)\rfloor{}\rrbracket & = &
\neg\lfloor\forall{}x.P(x)\rfloor\lor\lfloor{}P(t)\rfloor &
(\gamma_{\forall\mathrm{inst}})\\\\

\llbracket\neg\lfloor\exists{}x.P(x)\rfloor{}\rrbracket & = &
\lfloor\exists{}x.P(x)\rfloor\lor\neg\lfloor{}P(t)\rfloor &
(\gamma_{\neg\exists\mathrm{inst}})
\end{array}$
\end{center}}
\caption{Rules of Tableau Theory}
\label{fig:tabth}
\end{figure}

\subsection{The Rewriting Theory}
\label{sec:rew}

A rewriting theory allows us to introduce a computational behavior to the SAT
solver. We aim to integrate rewriting in the broadest sense of the term as
proposed by deduction modulo theory. Deduction modulo theory~\cite{DA03} focuses
on the computational part of a theory, where axioms are transformed into rewrite
rules, which induces a congruence over propositions, and where reasoning is
performed modulo this congruence. In deduction modulo theory, this congruence is
then induced by a set of rewrite rules over both terms and propositions.

In the following, we borrow some of the notations and definitions
of~\cite{DA03}. We call $\mathrm{FV}$ the function that returns the set of
free variables of a term or a formula. A term rewrite rule is a pair of terms
denoted by $l\rew{}r$, where $\mathrm{FV}(r)\subseteq\mathrm{FV}(l)$. A
proposition rewrite rule is a pair of propositions denoted by $l\rew{}r$, where
$l$ is an atomic proposition and $r$ is an arbitrary proposition, and where
$\mathrm{FV}(r)\subseteq\mathrm{FV}(l)$. A class rewrite system is a pair of
rewrite systems, denoted by $\mathcal{RE}$, consisting of $\mathcal{R}$, a set
of proposition rewrite rules, and $\mathcal{E}$, a set of term rewrite rules.

Given a class rewrite system $\mathcal{RE}$, the relations $=_\mathcal{E}$ and
$=_\mathcal{RE}$ are the congruences generated respectively by the sets
$\mathcal{E}$ and $\mathcal{R}\cup\mathcal{E}$. In the following, we use the
standard concepts of subterm and term replacement: given an occurrence $\omega$
in a proposition $P$, we write $P_{|\omega}$ for the term or proposition at
$\omega$, and $P[t]_\omega$ for the proposition obtained by replacing
$P_{|\omega}$ by $t$ in $P$ at $\omega$. Given a class rewrite system
$\mathcal{RE}$, the proposition $P$ $\mathcal{RE}$-rewrites to $P'$, denoted by
$P\rew_\mathcal{RE}P'$, if $P=_\mathcal{E}Q$, $Q_{|\omega}=\sigma(l)$, and
$P'=_\mathcal{E}Q[\sigma(r)]_\omega$, for some rule $l\rew{}r\in\mathcal{R}$,
some proposition $Q$, some occurrence $\omega$ in $Q$, and some substitution
$\sigma$.

The relation $=_\mathcal{RE}$ is not decidable in general, but there are some
cases where this relation is decidable depending on the class rewrite system
$\mathcal{RE}$ and the rewrite relation $\rew_\mathcal{RE}$. In particular, if
the rewrite relation $\rew_\mathcal{RE}$ is confluent and (weakly) terminating,
then the relation $=_\mathcal{RE}$ is decidable.

The rewriting theory is integrated into the SAT solver in a similar way as for
the tableau theory. Whenever a literal is propagated or decided, we generate a
clause introducing a new formula that is congruent to the formula in the box of
the literal. More precisely, given a literal $\lfloor{}P\rfloor$, where $P$ is a
formula, and a formula $P'$ such that $P=_\mathcal{RE}P'$, we generate the
following clause:

$$\left(\bigvee_{(l,r)\in\mathcal{R}}
\neg\lfloor\forall{}\vec{x}.l\Leftrightarrow{}r\rfloor\right)\lor
\left(\bigvee_{(l,r)\in\mathcal{E}}\neg\lfloor\forall{}\vec{x}.l=r\rfloor\right)\lor
\lfloor{}P\Leftrightarrow{}P'\rfloor$$

where $\vec{x}=\mathrm{FV}(l)\cup\mathrm{FV}(r)$.

It should be noted that in usual SMT solvers, rewriting can be emulated by means
of triggers that are actually the left-hand side members of the class rewrite
system $\mathcal{RE}$ introduced above. But in our rewriting theory, we can
generate the formula where rewriting setps have been done, while triggers can just
generate bindings, i.e. instances of the rewrite rules, which are used later to
relate the initial and rewritten formulas. Moreover, in our case, we can perform
several rewritings at once, while a trigger can only emulate one rewriting at a
time.

Let us illustrate the use of the rewriting theory by means of an example in set
theory. Let us prove that
$(\forall{}s,t.s\subseteq{}t\Leftrightarrow{}\forall{}x.x\in{}s\Rightarrow{}
x\in{}t) \Rightarrow{} {}a\subseteq{}a$, where $a$ is a constant. The proof is given
in Fig.~\ref{fig:rew}, with the state of the solver represented as $T \parallel S$,
with $T$ the trail of the SMT, in chronological order left to right, and $S$ the set
of clauses to be satisfied by the solver.

\begin{figure}[t!]
\parbox{\textwidth}
{\small
\begin{center}
$\begin{array}{lcl}
\emptyset\parallel
\neg{}A & \longrightarrow & (\mathrm{unit~prop})\\

  \boldsymbol{\neg{}A}\parallel{}\mathcal{C}_1 = \neg{}A & \longrightarrow &
(\mathrm{Tableaux})\\

\neg{}A\parallel
\mathcal{C}_1,\boldsymbol{\mathcal{C}_2 = A\lor{}B},\boldsymbol{\mathcal{C}_3 = A\lor\neg{}C}
& \longrightarrow & (\mathrm{unit~prop})\times{}2\\

\neg{}A,\boldsymbol{B},\boldsymbol{\neg{}C}\parallel
\mathcal{C}_1, \mathcal{C}_2, \mathcal{C}_3
& \longrightarrow & (\mathrm{Rewriting})\\

\neg{}A,B,\neg{}C\parallel
\mathcal{C}_1,\mathcal{C}_2,\mathcal{C}_3, \boldsymbol{\mathcal{C}_4=\neg{}B\lor{}D}
& \longrightarrow & (\mathrm{unit~prop})\\

\neg{}A,B,\neg{}C,\boldsymbol{D}\parallel
\mathcal{C}_1,\mathcal{C}_2,\mathcal{C}_3,\mathcal{C}_4
& \longrightarrow & (\mathrm{Tableaux})\\

\neg{}A,B,\neg{}C,D\parallel
\mathcal{C}_1,\mathcal{C}_2,\mathcal{C}_3,\mathcal{C}_4,
\boldsymbol{\mathcal{C}_5=\neg{}D\lor{}E},\boldsymbol{\mathcal{C}_6=\neg{}D\lor{}F}
& \longrightarrow & (\mathrm{unit~prop})\times{}2\\

\neg{}A,B,\neg{}C,D,\boldsymbol{E},\boldsymbol{F}\parallel
\mathcal{C}_1,\mathcal{C}_2,\mathcal{C}_3,\mathcal{C}_4, \mathcal{C}_5, \mathcal{C}_6
& \longrightarrow & (\mathrm{Tableaux})\\

\neg{}A,B,\neg{}C,D,E,F\parallel
\mathcal{C}_1,\mathcal{C}_2,\mathcal{C}_3,\mathcal{C}_4, \mathcal{C}_5, \mathcal{C}_6,
  \boldsymbol{\mathcal{C}_7=\neg{}F\lor\neg{}G\lor{}C}
& \longrightarrow & (\mathrm{unit~prop})\\

\neg{}A,B,\neg{}C,D,E,F,\boldsymbol{\neg{}G}\parallel
\mathcal{C}_1,\mathcal{C}_2,\mathcal{C}_3,\mathcal{C}_4, \mathcal{C}_5, \mathcal{C}_6, \mathcal{C}_7
& \longrightarrow & (\mathrm{Tableaux})\\

\neg{}A,B,\neg{}C,D,E,F,\neg{}G\parallel
\mathcal{C}_1,\mathcal{C}_2,\mathcal{C}_3,\mathcal{C}_4, \mathcal{C}_5, \mathcal{C}_6, \mathcal{C}_7,
  \boldsymbol{\mathcal{C}_8=G\lor\neg{}H}
& \longrightarrow & (\mathrm{unit~prop})\\

\neg{}A,B,\neg{}C,D,E,F,\neg{}G,\boldsymbol{\neg{}H}\parallel
\mathcal{C}_1,\mathcal{C}_2,\mathcal{C}_3,\mathcal{C}_4, \mathcal{C}_5, \mathcal{C}_6, \mathcal{C}_7, \mathcal{C}_8
& \longrightarrow & (\mathrm{Tableaux})\\

\neg{}A,B,\neg{}C,D,E,F,\neg{}G,\neg{}H\parallel
\mathcal{C}_1,\mathcal{C}_2,\mathcal{C}_3,\mathcal{C}_4, \mathcal{C}_5, \mathcal{C}_6, \mathcal{C}_7, \mathcal{C}_8, \\
~~~~\boldsymbol{\mathcal{C}_9=H\lor{}I},\boldsymbol{\mathcal{C}_{10}=H\lor\neg{}I}
& \longrightarrow & (\mathrm{unit~prop})\\

\neg{}A,B,\neg{}C,D,E,F,\neg{}G,\neg{}H,\boldsymbol{I}\parallel
\mathcal{C}_1,\mathcal{C}_2,\mathcal{C}_3,\mathcal{C}_4, \mathcal{C}_5, \mathcal{C}_6, \mathcal{C}_7, \mathcal{C}_8, \mathcal{C}_9, \mathcal{C}_{10}
& \longrightarrow & (\mathrm{unsat})\\

\boldsymbol{\mathrm{unsat}}
\end{array}$
\end{center}
\begin{flushleft}
$\begin{array}{l}
\mbox{where:}\\
\begin{array}{ll}
\multicolumn{2}{l}{
~~~~A\equiv\lfloor(\forall{}s,t.s\subseteq{}t\Leftrightarrow{}
\forall{}x.x\in{}s\Rightarrow{}x\in{}t)\Rightarrow{}a\subseteq{}a\rfloor}\\
~~~~B\equiv\lfloor\forall{}s,t.s\subseteq{}t\Leftrightarrow{}
\forall{}x.x\in{}s\Rightarrow{}x\in{}t\rfloor &
~~~~C\equiv\lfloor{}a\subseteq{}a\rfloor\\
~~~~D\equiv\lfloor{}a\subseteq{}a\Leftrightarrow
\forall{}x.x\in{}a\Rightarrow{}x\in{}a\rfloor &
~~~~E\equiv\lfloor{}a\subseteq{}a\Rightarrow
\forall{}x.x\in{}a\Rightarrow{}x\in{}a\rfloor\\
~~~~F\equiv\lfloor(\forall{}x.x\in{}a\Rightarrow{}x\in{}a)
\Rightarrow{}a\subseteq{}a\rfloor &
~~~~G\equiv\lfloor\forall{}x.x\in{}a\Rightarrow{}x\in{}a\rfloor\\
~~~~H\equiv\lfloor\epsilon_x\in{}a\Rightarrow\epsilon_x\in{}a\rfloor &
~~~~I\equiv\lfloor\epsilon_x\in{}a\rfloor
\end{array}\\\\
\mbox{with: }\epsilon_x=\epsilon(x).\neg(x\in{}a\Rightarrow{}x\in{}a)
\end{array}$
\end{flushleft}}
\caption{Example of Proof Using the Rewriting Theory}
\label{fig:rew}
\end{figure}
