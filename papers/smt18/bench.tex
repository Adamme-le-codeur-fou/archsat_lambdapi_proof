% $Id$

\section{Implementation and Experimental Results}
\label{sec:bench}

In this section, we briefly describe the implementation of our approach
introduced previously, and present some experimental results obtained by running
this implementation over a benchmark of problems in the \bmth{} set theory.

The algorithms described in this paper are implemented in the \archsat{}
automated theorem prover\footnote{Available at:
\url{https://gforge.inria.fr/projects/archsat}.}. It relies on the
\msat{}~\cite{GB17} library, derived from the \altergoz{} tool, which is a
generic library for building automated deduction tools based on SAT solvers.
\archsat{} (as well as \msat{}) is written in \ocaml{}. \archsat{} natively
supports polymorphic types as described in~\cite{BP13}.

\subsection{Experimental Results}

As a framework to test our tool, we consider the set theory of the \bmth{}
method~\cite{B-Book}. This method is supported by some tool sets, such as
\atelierb{}, which are used in industry to specify and build, by stepwise
refinements, software that is correct by design. This theory is suitable as it
can be easily turned into a theory that is compatible with deduction modulo
theory, i.e. where a large part of axioms can be turned into rewrite rules, and
for which the rewriting theory proposed previously in Subsec.~\ref{sec:rew}
works. Starting from the theory described in Chap.~2 of the
\bbook{}~\cite{B-Book}, we therefore transform whenever possible the axioms and
definitions into rewrite rules. The resulting theory has been introduced
in~\cite{BA15}. As can be seen, the proposed theory is typed, using first order
logic extended to polymorphic types à la ML, through a type system in the spirit
of~\cite{BP13}. This extension to polymorphic types offers more flexibility, and
in particular allows us to deal with theories that rely on elaborate type
systems, like the \bmth{} set theory (see Chap.~2 of the
\bbook{}~\cite{B-Book}).

To test \archsat{} in this theory, we consider a set of 319~lemmas coming from
Chap.~2 of the \bbook{}~\cite{B-Book}\footnote{Available at:
\url{https://github.com/delahayd/bset}}.  These lemmas are properties of
various difficulty regarding the set constructs introduced by the \bmth{}
method. It should be noted that these constructs and notations are, for a large
part of them, specific to the \bmth{} method, as they are used for the modeling
of industrial projects, and are not necessarily standard in set theory.

As tools, we consider \archsat{} (development version\footnote{\git{}
branch smt18.}). We also include other automated theorem provers, able to
deal with first order logic with polymorphic types and rewriting natively. In
particular, we consider \zenm{} (version~0.4.2), a tableau-based prover that
is an extension of \zenon{} to deduction modulo theory. To show the impact of
rewriting on the results, we also include the \altergo{} SMT solver
(version~1.01). It would have been possible to also consider provers dealing
with pure first order logic and encode the polymorphic layer. But preliminary
tests have been conducted and very low results have been obtained even for the
best state-of-the-art provers (we have considered \e{} and \cvc{} in
particular), which indicates that polymorphism encoding adds a lot of noise
in proof search and is not effective in practice.

The experiment was run on an \intel{}~3.50~GHz computer, with a timeout of 90s
(beyond this timeout, results do not change) and a memory limit of 1~GiB. The
results are summarized in Tab.~\ref{tab:bench}. In these results, we observe
that \archsat{} obtains better results, in terms of proved problems, than
\zenm{} and \altergo{}, which tends to show the effectiveness of our approach in
practice. Looking at the cumulative times, \altergo{} is not really faster than
\archsat{}, which take more time to find few more difficult
problems: with a timeout of 3s, \archsat{} finds 260 proofs in 16.61~s,
while \altergo{} obtains the same results.

\setlength{\tabcolsep}{3pt}
\renewcommand{\arraystretch}{1.2}
\newcolumntype{C}{>{\centering}X}

\begin{table}[t]
\begin{center}
\begin{tabularx}{\textwidth}{|X|C|C|C|}
\hline
\begin{tabular}{l}
319~Problems
\end{tabular} &
\begin{tabular}{c}
\archsat{}
\end{tabular} &
\begin{tabular}{c}
\zenm{}
\end{tabular} &
\begin{tabular}{c}
\altergo
\end{tabular}\tabularnewline
\hline
\begin{tabular}{l}
Proofs
\end{tabular} &
\begin{tabular}{c}
272
\end{tabular} &
\begin{tabular}{c}
138
\end{tabular} &
\begin{tabular}{c}
232
\end{tabular}\tabularnewline
\hline
\begin{tabular}{l}
Rate
\end{tabular} &
\begin{tabular}{c}
85.3\%
\end{tabular} &
\begin{tabular}{c}
43.3\%
\end{tabular} &
\begin{tabular}{c}
72.7\%
\end{tabular}\tabularnewline
\hline
\begin{tabular}{l}
Total time \small{(s)}
\end{tabular} &
\begin{tabular}{c}
268.69
\end{tabular} &
\begin{tabular}{c}
2.86
\end{tabular} &
\begin{tabular}{c}
8.42
\end{tabular}\tabularnewline
\hline
\end{tabularx}
\end{center}
\caption{Experimental Results over the \bmth{} Set Theory Benchmark}
\label{tab:bench}
\end{table}

\renewcommand{\arraystretch}{1}
