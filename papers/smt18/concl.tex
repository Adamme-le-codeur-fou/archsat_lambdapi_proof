% $Id$

\section{Conclusion}

We have described the architecture of \archsat{}, an automated theorem prover
that combines a SMT solver with tableau calculus and rewriting. Compared to
several other tools, \archsat{} appears quite effective in practice, as shown by
some experimental results obtained by running our implementation over a
benchmark of problems in the \bmth{} set theory.

As perspectives, we plan to realize more tests of \archsat{} over other theories
where a large part of these theories can be turned into rewrite rules. In
particular, a regular trigger mechanism has been also implemented in \archsat{}
and can be used to deal with conditional rewriting (the instantiation is delayed
and performed once the condition has been evaluated to true). This feature
should open up a range of new perspectives on the theories that our approach
could handle. We also aim to apply our tool to the benchmark of the \bware{}
project~\cite{BWare}, which consists of a large collection of proof obligations
coming from the development of industrial applications using the \bmth{} method.
This collection gathers about 13,000~problems, and should allow us to understand
to what extent our tool scales up, though it requires to extend \archsat{} to
handle arithmetic, which is why it has not been tested yet.
