% $Id$

\section{Introduction}

Over the last past few
years, SMT solvers have appeared as very efficient tools to reason over some
well identified theories (equality, uninterpreted functions, linear arithmetic,
arrays, etc.), and have allowed us to bring SAT solving toward first order logic.
Although modern SMT solvers support first order logic, most of them use
heuristic quantifier instantiation for incorporating quantifier reasoning with
ground decision procedures. This mechanism is relatively effective in some cases
in practice, but it is not refutationally complete for first order logic. Hints
(triggers) are usually required, and it is sensitive to the syntactic structure
of the formula, so that it fails to prove formulas that can be easily discharged
by provers based on more traditional first order proof search methods (tableaux,
resolution, etc.).

In this paper, we propose to improve first order proof search by introducing
rewriting into the SAT solver as a regular SMT theory, along the lines of
deduction modulo theory. Deduction modulo theory~\cite{DA03} focuses on the
computational part of a theory, where axioms are transformed into rewrite rules,
which induces a congruence over propositions, and where reasoning is performed
modulo this congruence. In deduction modulo theory, this congruence is then
induced by a set of rewrite rules over both terms and propositions. In our
proposal, this congruence is used a first time to speed up ground reasoning
by computing normal forms for terms, but this still yields an incomplete
algorithm.

We thus propose to overcome the problem of completeness for first
order logic by using tableau calculus as an SMT theory. The tableau calculus
rules are used to unfold propositional content into clauses while atomic
formulas are handled using satisfiability decision procedures as in
regular SMT solvers. To deal with quantified first order formulas, we use
metavariables and perform rigid unification modulo equalities and modulo rewriting,
for which we introduce an algorithm based on superposition, but where all clauses
contain a single atomic formula.

Our approach provides several advantages compared to usual SMT solving and first
order proof search methods. First, we benefit from the efficiency of a SAT
solver core together with a complete method of instantiation (when a
propositional model is found, we try to find a conflict between two literals by
unification). Second, it should be noted that our approach requires no change in
the architecture of the SMT solver, since the tableau calculus and rewriting are
seen as regular theories. Finally, no preliminary Skolemization and Conjunctive
Normal Form (CNF) transformation is required. This transformation is performed
lazily by applying the tableau rules progressively when a literal is propagated
or decided. This makes the production of genuine output proofs easier, contrary
to the usual approach, where the Skolemization/CNF translation is realized at
the beginning and externalized with respect to the proof search.

Our proposal combining SMT solving with tableau calculus and rewriting has been
implemented and the corresponding tool is called \archsat{}. This tool is able
to deal with first order logic extended to polymorphic types à la ML, through a
type system in the spirit of~\cite{BP13}. To test this tool, we propose a
benchmark in the framework of the set theory of the \bmth{}
method~\cite{B-Book}. This theory~\cite{BA15} has been expressed using first
order logic extended to polymorphic types and turned into a theory that is
compatible with deduction modulo theory, i.e. where a large part of axioms has
been turned into rewrite rules. The benchmark itself gathers 319~lemmas coming
from Chap.~2 of the \bbook{}~\cite{B-Book}.

The paper is organized as follows: in Sec.~\ref{sec:smt}, we introduce the tableau
and rewriting theories; we then describe, in Sec.~\ref{sec:super}, our mechanism
of equational reasoning by means of rigid unit superposition; finally, in
Sec.~\ref{sec:bench}, we present some experimental results obtained by running
our implementation over a benchmark of problems in the \bmth{} set theory.
