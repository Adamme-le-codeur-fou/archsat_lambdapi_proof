% $Id$

\documentclass[orivec]{llncs}

\usepackage[utf8]{inputenc}
\usepackage[T1]{fontenc}
\usepackage{makeidx}
\usepackage[hidelinks]{hyperref}
\usepackage{bussproofs}
\usepackage{amsmath,amsfonts,amssymb}
\usepackage{stmaryrd}
\usepackage{tabularx}
\usepackage{pgf,tikz}
\usepackage{url}
\usepackage{float}
\usepackage{color}
\usepackage{cite}
\usepackage{booktabs} % for tabular
\floatstyle{boxed}
\restylefloat{figure}
\usepackage{relsize}

% $Id$

\def\ai{\textsf{Aesthetic~Integration}}
\def\alloy{\textsf{Alloy}}
\def\altergo{\textsf{Alt-Ergo}}
\def\altergoz{\textsf{Alt-Ergo~Zero}}
\def\archsat{\textsf{ArchSAT}}
\def\atelierb{\textsf{Atelier~B}}
\def\bbook{\textsf{B}-Book}
\def\bmth{\textsf{B}}
\def\bware{\textsf{BWare}}
\def\cvc{\textsf{CVC4}}
\def\e{\textsf{E}}
\def\ens{\textsf{ENS Paris-Saclay}}
\def\git{\textsf{Git}}
\def\inria{\textsf{Inria}}
\def\intel{\textsf{Intel~Xeon~E5-1650~v3}}
\def\iproverm{\textsf{iProver~Modulo}}
\def\lirmm{\textsf{LIRMM}}
\def\lsv{\textsf{LSV}}
\def\msat{\textsf{mSAT}}
\def\ocaml{\textsf{OCaml}}
\def\princess{\textsf{Princess}}
\def\um{\textsf{Université de Montpellier}}
\def\vdm{\textsf{VDM}}
\def\satallax{\textsf{Satallax}}
\def\zenm{\textsf{Zenon~Modulo}}
\def\zenon{\textsf{Zenon}}
\def\zipper{\textsf{Zipperposition}}
\def\znot{\textsf{Z}}

\EnableBpAbbreviations{}
\newcommand{\UICm}[1]{\UIC{$#1$}}
\newcommand{\AXCm}[1]{\AXC{$#1$}}
\newcommand{\BICm}[1]{\BIC{$#1$}}
\newcommand{\TICm}[1]{\TIC{$#1$}}
\newcommand{\RLm}[1]{\RL{$#1$}}

\newcommand\clauseWithSubst[2]{\ensuremath{#1 ~|~ #2}}
\newcommand\renameVarsSymb{\ensuremath{\textsf{rename}}}
\newcommand\renameVars[1]{\ensuremath{\renameVarsSymb}(#1)}
\newcommand\mapVar[2]{\ensuremath{ #1 \mapsto #2 }}

\def\rew{\longrightarrow}
\def\arg{\mbox{-}}
\def\type{\textsf{Type}}
\def\omicron{o}
\newcommand\set[1]{\mathsf{set}(#1)}
\newcommand\tuple[2]{\mathsf{tup}(#1,#2)}
\def\plus{\raisebox{.22\height}{\scalebox{.6}{\pmb{+}}}}
\def\fctpart{~{\kern+.2em\mapstochar{}\!\!\!\!\!\rightarrow}~}
\def\injpart{~~{\mapstochar{}\!\!\!\!\!\!\rightarrowtail}~}
\def\surpart{~{\kern+.2em\mapstochar{}\!\!\!\!\!\twoheadrightarrow}~}
\def\bijpart{~{\kern+.1em\mapstochar{}\!\!\!\!\!\rightarrowtail\kern+.03em
\!\!\!\!\!\!\twoheadrightarrow}~}
\def\bijtotal{~{\kern-.1em\rightarrowtail\kern+.03em\!\!\!\!\!\!
\twoheadrightarrow}~}
\def\override{~~{\plus{}\mkern-24mu<}~}
\def\substleft{~{-\mkern-14mu\lhd}~}
\def\substright{~{-\mkern-14mu\rhd}~}

\newlength\memparindent
\setlength{\memparindent}{\parindent}
\newcommand{\myindent}{\hspace{\memparindent}}

\newcommand\todo[1]{\textcolor{red}{#1}}


\begin{document}

\title{SAT Solving Modulo Tableau\break{}and Rewriting Theories}
\titlerunning{SAT Solving Modulo Tableau and Rewriting Theories}

\author{Guillaume~Bury\inst{1} \and Simon~Cruanes\inst{2} \and
David~Delahaye\inst{3}}
\authorrunning{Guillaume~Bury, Simon~Cruanes, and David~Delahaye}
\tocauthor{Guillaume~Bury, Simon~Cruanes, and David~Delahaye}

\institute{\inria{}/\lsv{}/\ens{}, Cachan, France\\
\email{Guillaume.Bury@inria.fr} \and
\ai{}, Austin (Texas), USA\\
\email{simon@aestheticintegration.com} \and
\lirmm{}/\um{}, Montpellier, France\\
\email{David.Delahaye@lirmm.fr}}

\maketitle

\begin{abstract}
We propose an automated theorem prover that combines a SAT solver with tableau
calculus and rewriting. Tableau inference rules are used to unfold propositional
content into clauses while atomic formulas are handled using satisfiability
decision procedures as in SMT solvers. To deal with quantified first order
formulas, we use metavariables and perform rigid unification modulo equalities
and rewriting, for which we introduce an algorithm based on superposition, but
where all clauses contain a single atomic formula. Rewriting is introduced along
the lines of deduction modulo theory, where axioms are turned into rewrite rules
over both propositions and terms. Finally, we assess our approach over a
benchmark of lemmas in the set theory of the \bmth{} method.

\keywords{SAT, SMT, Tableaux, Rewriting, Superposition}
\end{abstract}

% $Id$

\section{Introduction}

These days, Boolean satisfiability (SAT) solvers are undeniably quite reliable
tools despite the worst-case exponential run time of all known
algorithms. Modern SAT solvers provide a ``black-box'' procedure that can often
solve hard structured problems with over a million variables and several million
constraints. These tools are therefore used in many applications outside the
scope of knowledge representation and reasoning, which nonetheless was their
initial domain of application. The underlying representational formalism of SAT
solvers is propositional logic, and even if a given problem is not a pure
propositional problem, it is not unusual to encode it so that it can be viewed
as a propositional reasoning task. This encoding may be complex and possibly
costly, but the effectiveness of SAT solvers generally compensates for the cost
of this encoding.

This growing interest for SAT solvers has steered them toward a more first order
approach, with the possibility to deal with first order literals and reason
modulo theories. This resulted in the design of Satisfiability Modulo Theories
(SMT) solvers, which are built on top of SAT solvers. Over the last past few
years, SMT solvers have appeared as very efficient tools to reason over some
well identified theories (equality, uninterpreted functions, linear arithmetic,
arrays, etc.), and have allowed us to drag SAT solving toward first order logic.
Although modern SMT solvers supports first order logic, most of them use
heuristic quantifier instantiation for incorporating quantifier reasoning with
ground decision procedures. This mechanism is relatively effective in some cases
in practice, but it is not refutationally complete for first order logic. Hints
(triggers) are usually required, and it is sensitive to the syntactic structure
of the formula, so that it fails to prove formulas that can be easily discharged
by provers based on more traditional first order proof search methods (tableaux,
resolution, etc.).

In this paper, we propose to overcome this problem of completeness for first
logic in SMT solving by combining a SAT solver core with tableau calculus. The
tableau calculus is introduced as a regular theory of SMT solving, and the rules
are used to unfold propositional content into clauses while atomic formulas are
handled using satisfiability decision procedures as usually done in SMT solvers.
In order to manage quantified first order formulas, we use metavariables and
perform rigid unification modulo equalities, for which we introduce an algorithm
based on superposition, but where all clauses contain a single atomic formula.

We also propose to improve proof search by introducing rewriting to the SAT
solver as a regular SMT theory and along the lines of deduction modulo
theory. Deduction modulo theory~\cite{DA03} focuses on the computational part of
a theory, where axioms are transformed into rewrite rules, which induces a
congruence over propositions, and where reasoning is performed modulo this
congruence. In deduction modulo theory, this congruence is then induced by a set
of rewrite rules over both propositions and terms.

Our approach provides several advantages compared to usual SMT solving and first
order proof search methods. First, we benefit from the efficiency of a SAT
solver core together with a complete method of instantiation (when a
propositional model is found, we try to find a conflict between two literals by
unification). Second, it should be noted that our approach requires no change in
the architecture of the SMT solver, since the tableau calculus and rewriting are
seen as regular theories. Finally, no preliminary Skolemization and Conjunctive
Normal Form (CNF) transformation is required. This transformation is performed
lazily by applying the tableau rules progressively when a literal is propagated
or decided.  This makes the production of genuine output proofs easier, contrary
to the usual approach, where the Skolemization/CNF translation is realized at
the beginning and externalized with respect to the proof search.

Our proposal combining SAT solving with tableau calculus and rewriting has been
implemented and the corresponding tool is called \archsat{}. This tool is able
to deal with first order logic extended to polymorphic types à la ML, through a
type system in the spirit of~\cite{BP13}. To test this tool, we propose a
benchmark in the framework of the set theory of the \bmth{}
method~\cite{B-Book}. This theory~\cite{BA15} has been expressed using first
order logic extended to polymorphic types and turned into a theory that is
compatible with deduction modulo theory, i.e. where a large part of axioms has
been turned into rewrite rules. The benchmark itself gathers 319~lemmas coming
from Chap.~2 of the \bbook{}~\cite{B-Book}.

The paper is organized as follows: in Sec.~\ref{sec:smt}, we recall the classic
architecture of SMT solvers and we introduce the tableau and rewriting theories;
we then describe, in Sec.~\ref{sec:rusuper}, our mechanism of equational
reasoning by means of rigid unit superposition; finally, in
Sec.~\ref{sec:bench}, we present some experimental results obtained by running
our implementation over a benchmark of lemmas in the \bmth{} set theory, and
also provide related work in Sec.~\ref{sec:rwork}.

% $Id$

\section{SAT Solving Modulo Tableau and Rewriting Theories}
\label{sec:smt}

In this section, we recall the classic architecture of SMT solvers and we then
introduce the tableau and rewriting theories.

Compared to genuine tableau automated theorem provers, like \princess{} or
\zenon{} for example, our approach has the benefit of being versatile since the
tableau rules are actually integrated as a regular SMT theory. This way, the
tableau rules can be easily combined with other theories, such as equality logic
with uninterpreted functions or arithmetic. The way we integrate the tableau
rules into the SAT solver (mainly by boxing/unboxing first order formulas) is
close to what is done in the \satallax{} tool~\cite{CEB12}. The difference
resides in the fact that we are in a pure first order framework, which has
significant consequences in the management of quantifiers and unification in
particular (see Sec.~\ref{sec:super}).

Regarding the integration of rewriting, automated theorem provers rely on
several solutions (superposition rule for first order provers, triggers for SMT
solvers, etc.). But deduction modulo theory~\cite{DA03} is probably the most
general approach, where a theory can be partly turned into a set of rewrite
rules over both terms and propositions. Several proof search methods have been
extended to deduction modulo theory, resulting in tools such as \iproverm{} and
\zenm{}. This paper can be seen as a continuation of these previous experiments
adapted to the framework of SMT solving.

\subsection{SAT Solving Modulo Theories}

In this subsection, we recall the classic architecture of SMT
solvers~\cite{BEA06}. We introduce $\mathcal{T}$ and $\mathcal{F}$ respectively,
the sets of first order terms and formulas over the signature
$\mathcal{S}=(\mathcal{S}_\mathcal{F},\mathcal{S}_\mathcal{P})$, where
$\mathcal{S}_\mathcal{F}$ is the set of function symbols, and
$\mathcal{S}_\mathcal{P}$, the set of predicate symbols, such that
$\mathcal{S}_\mathcal{F}\cap\mathcal{S}_\mathcal{P}=\emptyset$. The set
$\mathcal{T}$ is extended with two kinds of terms specific to tableau proof
search. First are $\epsilon{}$-terms (used instead of Skolemization) of the form
$\epsilon(x).P(x)$, where $P(x)$ is a formula, and which means some $x$ that
satisfies $P(x)$, if it exists. And second, metavariables (often named free variables in
the tableau-related literature) of the form $X_P$, where $P$ is the formula that
introduces the metavariable, and which is either $\forall{}x.Q(x)$ or
$\neg\exists{}x.Q(x)$, with $Q(x)$ a formula.

A boxed formula is of the form $\lfloor{}P\rfloor$, where $P$ is a formula. A
boxed formula is called an atom, and a literal is either an atom, or the negation
of an atom. A literal is such that there is no negation on top of the boxed
formula (which means that $\lfloor\neg{}P\rfloor=\neg\lfloor{}P\rfloor$, and
$\neg\neg\lfloor{}P\rfloor=\lfloor{}P\rfloor$). A clause is a disjunction of
literals. It should be noted that SAT solving usually reasons over sets of
clauses composed of first order literals; here, a literal is a first order
formula (possibly with quantifiers), which requires to box formulas to get a
regular SAT solving problem where boxed formulas are propositional variables.

An assignment of the boxed formulas is a partial function that assigns to boxed
formulas an element of $\{\top,\bot\}$ (values of the Boolean algebra). Given a
clause $C$, we say that $C$ is valid and write $\models{}C$, if $C$ evaluates to
$\top$ for every assignment. Given a set of clauses $S$, we say that $S$ is
valid and write $\models{}S$, if each clause of $S$ is valid. Given two sets of
clauses $S$ and $S'$, we say that $S$ implies $S'$ and write $S\models{}S'$ if,
for every assignment, if each clause of $S$ evaluates to $\top$ then each clause
of $S'$ evaluates to $\top$. If $S$ is a set of literals, then we say that $S$
is a model of $S'$.

A theory $T$ is a set of formulas over a given signature $\mathcal{S}$, which
are called axioms of the theory $T$. A clause $C$ is a tautology in the theory
$T$, written $\models_TC$, if the axioms of $T$ imply the formula $C'$, where
$C'$ is the clause $C$ where all the boxed formulas have been unboxed. A set of
clauses $S$ is a tautology in the theory $T$ if each clause of $S$ is a
tautology in $T$. Given two sets of clauses $M$ and $S$, we say that $M$ is a
model of $S$ modulo the theory $T$ and write $M\models_TS$, if $M$ is a model of
$S$ and $M$ is a tautology in $T$.

A SAT solver modulo theory is characterized by a theory $T$ and an
internal state of the form $M\parallel{}S$, where $M$ is an ordered list of
literals (if $M=l_1,l_2,\ldots,l_n$ then $l_1<l_2<\ldots<l_n$) representing
decisions and propagations in chronological order, and $S$ a set of
clauses. Intuitively, $M$ represents the model of $S$, which will be built
progressively (initially, we have $M=\emptyset$) by the application of a set of
rules over the internal state of the solver. This set of rules is presented in
Fig.~\ref{fig:smt}. Each time a literal is propagated (rule
``$\mathrm{unit~prop}$'') or decided (rule ``$\mathrm{decide}$''), it is given
to each theory, which may generate new clauses (rule ``$\mathrm{learn}$''). If a
clause is unvalidated by the model, then it is possible to backtrack the
decision over a given literal of the model if it exists (rule
``$\mathrm{backjump}$''). Two cases wrt final states can be distinguished.
The first one is ``$\mathrm{unsat}$'', which means that $S$ is
unsatisfiable (rule ``$\mathrm{unsat}$''). The second one is $M\parallel{}S$,
where $M$ is a model of $S$ and a tautology the theory $T$. In this
last case, if $M$ is not a tautology in a theory $T$, this theory generates a
set of conflict clauses (rule ``$\mathrm{learn}$'').

\begin{figure}[t]
\parbox{\textwidth}
{\small
\begin{center}
$\begin{array}{l@{\hspace{0.4cm}}l@{\hspace{0.4cm}}l}
M\parallel{}S,C\lor{}l\longrightarrow{}M,l\parallel{}S,C\lor{}l &
  (\mathrm{unit~prop}) & \mbox{if }l\mbox{ undefined in }{}M\mbox{ and }M\models{}\neg{}C\\\\

M\parallel{}S\longrightarrow{}M,l^\mathrm{d}\parallel{}S & (\mathrm{decide}) &
\mbox{if }l\not\in{}M\mbox{, and }l\in{}S\mbox{ or }\neg{}l\in{}S\\\\

M\parallel{}S,C\longrightarrow\mathrm{unsat} & (\mathrm{unsat}) &
\mbox{if }M'\models{}\neg{}C\mbox{ s.t. }M'\subseteq{}M\\
&& \mbox{and there is no }l^\mathrm{d}\leq{}l'\mbox{ in }M\\
&& \mbox{for all }l'\in{}M'\\\\
 
M,l^\mathrm{d},M'\parallel{}S,C\longrightarrow{}M,l'\parallel{}S,C &
(\mathrm{backjump}) & \mbox{if }M,l^\mathrm{d},M'\models\neg{}C
\mbox{, and there is}\\
&& \mbox{some clause }C'\lor{}l' s.t.:\\
&& l'\not\in{}M\mbox{, and }l'\in{}S\mbox{ or }\neg{}l'\in{}S\\
&& \mbox{or }l'\in{}M,l^\mathrm{d},M'\mbox{ or }
\neg{}l'\in{}M,l^\mathrm{d},M',\\
&& \mbox{and }S,C\models{}C'\lor{}l'\mbox{, and }M\models\neg{}C'\\\\

M\parallel{}S\longrightarrow{}M\parallel{}S,S'
& (\mathrm{learn}) & \models_TS'\mbox{, where }T\mbox{ is a theory}
\end{array}$
\end{center}}
\caption{Rules of SAT Solving Modulo Theory}
\label{fig:smt}
\end{figure}

\subsection{The Tableau Theory}
\label{sec:tab}

In the SMT solver previously described, the tableau proof search method is
integrated as a regular theory. When a literal is propagated or decided, we
generate a set of clauses corresponding to the application of a tableau rule
depending on the logical connective at the root of the formula in the box of the
literal. More precisely, for a literal $l$, we generate the set of clauses
$\llbracket{}l\rrbracket$, where the function $\llbracket\cdot\rrbracket$ is
described by the rules of Fig.~\ref{fig:tabth}. When a literal is propagated or
decided, we use all the rules of Fig.~\ref{fig:tabth} except the instantiation
$\gamma$-rules (rules $\gamma_{\forall\mathrm{inst}}$ and
$\gamma_{\neg\exists\mathrm{inst}}$). It should be noted that we use the same
names for the rules as in tableau calculus ($\alpha$-rules, $\beta$-rules,
etc.), but there is no precedence between rules and therefore no priority in the
application of the rules contrary to the tableau proof search method (where
$\alpha$ rules are applied before $\beta$-rules, and so on).

When the SMT solver reaches a state $M\parallel{}S$, where $M$ is a model of
$S$, we look for a conflict in $M$ between two literals by unification and we
generate the clauses corresponding to the instantiation of the metavariables
using the result of the unification. More precisely, if there exist two literals
$l$ and $\neg{}l'$ in $M$ such that $l=\lfloor{}Q\rfloor$ and
$l'=\lfloor{}R\rfloor$, with $Q$ and $R$ two formulas, then for each binding
$(X_{\forall{}x.P(x)}\mapsto{}t)\in\sigma$ belonging to a
substitution\footnote{In practice, the implementation also splits and
reduce substitutions in order to only generate clauses corresponding to the
outermost meta-variables.}
$\sigma\in\mathrm{mgu}(Q,R)$ (resp.
$(X_{\neg\exists{}x.P(x)}\mapsto{}t)\in\sigma\in\mathrm{mgu}(Q,R)$)
we can generate the clauses $\llbracket\lfloor\forall{}x.P(x)\rfloor{}\rrbracket$
(resp. $\llbracket\neg\lfloor\exists{}x.P(x)\rfloor{}\rrbracket$) using the rule
$\gamma_{\forall\mathrm{inst}}$ (resp. $\gamma_{\neg\exists\mathrm{inst}}$) of
Fig.~\ref{fig:tabth}.

\begin{figure}[t]
\parbox{\textwidth}
{\small
\underline{Analytic Rules}
\begin{center}
$\begin{array}{lll@{\hspace{0.5cm}}l}
\llbracket\lfloor{}P\land{}Q\rfloor{}\rrbracket & = &
\neg\lfloor{}P\land{}Q\rfloor\lor\lfloor{}P\rfloor,
\neg\lfloor{}P\land{}Q\rfloor\lor\lfloor{}Q\rfloor & (\alpha_\land)\\\\

\llbracket\neg\lfloor{}P\land{}Q\rfloor{}\rrbracket & = &
\lfloor{}P\land{}Q\rfloor\lor\neg\lfloor{}P\rfloor\lor\neg\lfloor{}Q\rfloor &
(\beta_{\neg\land})\\\\

\llbracket{}\lfloor{}P\lor{}Q\rfloor{}\rrbracket & = &
\neg\lfloor{}P\lor{}Q\rfloor\lor\lfloor{}P\rfloor\lor\lfloor{}Q\rfloor &
(\beta_\lor)\\\\

\llbracket\neg\lfloor{}P\lor{}Q\rfloor{}\rrbracket & = &
\lfloor{}P\lor{}Q\rfloor\lor\neg\lfloor{}P\rfloor,
\lfloor{}P\lor{}Q\rfloor\lor\neg\lfloor{}Q\rfloor &
(\alpha_{\neg\lor})\\\\

\llbracket{}\lfloor{}P\Rightarrow{}Q\rfloor{}\rrbracket & = &
\neg\lfloor{}P\Rightarrow{}Q\rfloor\lor\neg\lfloor{}P\rfloor\lor
\lfloor{}Q\rfloor & (\beta_\Rightarrow)\\\\

\llbracket\neg\lfloor{}P\Rightarrow{}Q\rfloor{}\rrbracket & = &
\lfloor{}P\Rightarrow{}Q\rfloor\lor\lfloor{}P\rfloor,
\lfloor{}P\Rightarrow{}Q\rfloor\lor\neg\lfloor{}Q\rfloor &
(\alpha_{\neg\Rightarrow})\\\\

\llbracket{}\lfloor{}P\Leftrightarrow{}Q\rfloor{}\rrbracket & = &
\neg\lfloor{}P\Leftrightarrow{}Q\rfloor\lor\lfloor{}P\Rightarrow{}Q\rfloor,
\neg\lfloor{}P\Leftrightarrow{}Q\rfloor\lor\lfloor{}Q\Rightarrow{}P\rfloor &
(\beta_\Rightarrow)\\\\

\llbracket{}\neg\lfloor{}P\Leftrightarrow{}Q\rfloor{}\rrbracket &
= &
\lfloor{}P\Leftrightarrow{}Q\rfloor\lor\neg\lfloor{}P\Rightarrow{}Q\rfloor\lor
\neg\lfloor{}Q\Rightarrow{}P\rfloor & (\beta_{\neg\Rightarrow})
\end{array}$
\end{center}

\underline{$\delta$-Rules}
\begin{center}
$\begin{array}{lll@{\hspace{0.5cm}}l}
\llbracket\lfloor\exists{}x.P(x)\rfloor{}\rrbracket & = &
\neg{}\lfloor\exists{}x.P(x)\rfloor\lor\lfloor{}P(\epsilon(x).P(x))\rfloor &
(\delta_\exists)\\\\

\llbracket\neg\lfloor\forall{}x.P(x)\rfloor{}\rrbracket & = &
\lfloor\forall{}x.P(x)\rfloor\lor\neg\lfloor{}P(\epsilon(x).\neg{}P(x))\rfloor &
(\delta_{\neg\forall})
\end{array}$
\end{center}

\underline{$\gamma$-Rules}
\begin{center}
$\begin{array}{lll@{\hspace{0.5cm}}l}
\llbracket\lfloor\forall{}x.P(x)\rfloor{}\rrbracket & = &
\neg\lfloor\forall{}x.P(x)\rfloor\lor\lfloor{}P(X_{\forall{}x.P(x)})\rfloor &
(\gamma_{\forall{}M})\\\\

\llbracket\neg\lfloor\exists{}x.P(x)\rfloor{}\rrbracket & = &
\lfloor\exists{}x.P(x)\rfloor\lor
\neg\lfloor{}P(X_{\neg\exists{}x.P(x)})\rfloor &
(\gamma_{\neg\exists{}M})\\\\

\llbracket\lfloor\forall{}x.P(x)\rfloor{}\rrbracket & = &
\neg\lfloor\forall{}x.P(x)\rfloor\lor\lfloor{}P(t)\rfloor &
(\gamma_{\forall\mathrm{inst}})\\\\

\llbracket\neg\lfloor\exists{}x.P(x)\rfloor{}\rrbracket & = &
\lfloor\exists{}x.P(x)\rfloor\lor\neg\lfloor{}P(t)\rfloor &
(\gamma_{\neg\exists\mathrm{inst}})
\end{array}$
\end{center}}
\caption{Rules of Tableau Theory}
\label{fig:tabth}
\end{figure}

To show how this theory works, let us prove that
$\exists{}x.P(x)\Rightarrow{}P(a)\land{}P(b)$, where $P$ is a predicate symbol
and $a,b$ two constants. The SAT solver is initiated with the state
$\emptyset\parallel
\neg\lfloor\exists{}x.P(x)\Rightarrow{}P(a)\land{}P(b)\rfloor$, and the proof is
described in Fig.~\ref{fig:tab}, where $X$ is a shortcut for
$X_{\neg\exists{}x.P(x)\Rightarrow{}P(a)\land{}P(b)}$. It should be noted that
to do this proof in sequent calculus, a right contraction is necessary to
instantiate the formula twice (with $a$ and $b$). In Fig.~\ref{fig:tab}, it is
done by propagating the literal $\lfloor{}P(X)\rfloor{}$, which can provide as
many instantiations (by unification) as necessary.

\begin{figure}[t!]
\parbox{\textwidth}
{\small
\begin{center}
$\begin{array}{lcl}
\emptyset\parallel\neg{}A & \longrightarrow & (\mathrm{unit~prop})\\

\boldsymbol{\neg{}A}\parallel{}\neg{}A & \longrightarrow & (\mathrm{learn})\\

\neg{}A\parallel\neg{}A,\boldsymbol{A\lor\neg{}B} & \longrightarrow &
(\mathrm{unit~prop})\\

\neg{}A,\boldsymbol{\neg{}B}\parallel\neg{}A,A\lor\neg{}B & \longrightarrow &
(\mathrm{learn})\\

\neg{}A,\neg{}B\parallel\neg{}A,A\lor\neg{}B,\boldsymbol{B\lor{}C},
\boldsymbol{B\lor\neg{}D} & \longrightarrow & (\mathrm{unit~prop})\times{}2\\

\neg{}A,\neg{}B,\boldsymbol{C},\boldsymbol{\neg{}D}\parallel\neg{}A,
A\lor\neg{}B,B\lor{}C,B\lor\neg{}D & \longrightarrow & (\mathrm{learn})\\

\neg{}A,\neg{}B,C,\neg{}D\parallel\neg{}A,A\lor\neg{}B,B\lor{}C,B\lor\neg{}D,\\
~~~~\boldsymbol{D\lor\neg{}E\lor\neg{}F} & \longrightarrow & (\mathrm{decide})\\

\neg{}A,\neg{}B,C,\neg{}D,\boldsymbol{\neg{}E^d}\parallel
\neg{}A,A\lor\neg{}B,B\lor{}C, & \longrightarrow &
(\mathrm{learn})\\
~~~~B\lor\neg{}D,D\lor\neg{}E\lor\neg{}F && \{X\mapsto{}a\}=\mathrm{mgu}(C,E)\\

\neg{}A,\neg{}B,C,\neg{}D,\neg{}E^d\parallel
\neg{}A,A\lor\neg{}B,B\lor{}C,\\
~~~~B\lor\neg{}D,D\lor\neg{}E\lor\neg{}F,\boldsymbol{A\lor\neg{}G} &
\longrightarrow & (\mathrm{unit~prop})\\

\neg{}A,\neg{}B,C,\neg{}D,\neg{}E^d,\boldsymbol{\neg{}G}\parallel
\neg{}A,A\lor\neg{}B,B\lor{}C,\\
~~~~B\lor\neg{}D,D\lor\neg{}E\lor\neg{}F,A\lor\neg{}G & \longrightarrow &
(\mathrm{learn})\\

\neg{}A,\neg{}B,C,\neg{}D,\neg{}E^d,\neg{}G\parallel
\neg{}A,A\lor\neg{}B,B\lor{}C,\\
~~~~B\lor\neg{}D,D\lor\neg{}E\lor\neg{}F,A\lor\neg{}G,\boldsymbol{G\lor{}E},
\boldsymbol{G\lor\neg{}D} & \longrightarrow & (\mathrm{backjump})\\

\neg{}A,\neg{}B,C,\neg{}D,\boldsymbol{E}\parallel
\neg{}A,A\lor\neg{}B,B\lor{}C,\\
~~~~B\lor\neg{}D,D\lor\neg{}E\lor\neg{}F,A\lor\neg{}G,G\lor{}E,G\lor\neg{}D &
\longrightarrow & (\mathrm{unit~prop})\times{}2\\

\neg{}A,\neg{}B,C,\neg{}D,E,\boldsymbol{\neg{}F},\boldsymbol{\neg{}G}\parallel
\neg{}A,A\lor\neg{}B,B\lor{}C, & \longrightarrow & (\mathrm{learn})\\
~~~~B\lor\neg{}D,D\lor\neg{}E\lor\neg{}F,A\lor\neg{}G,G\lor{}E,G\lor\neg{}D &&
\{X\mapsto{}b\}=\mathrm{mgu}(C,F)\\

\neg{}A,\neg{}B,C,\neg{}D,E,\neg{}F,\neg{}G\parallel
\neg{}A,A\lor\neg{}B,B\lor{}C,\\
~~~~B\lor\neg{}D,D\lor\neg{}E\lor\neg{}F,A\lor\neg{}G,G\lor{}E,G\lor\neg{}D\\
~~~~\boldsymbol{A\lor\neg{}H} & \longrightarrow & (\mathrm{unit~prop})\\

\neg{}A,\neg{}B,C,\neg{}D,E,\neg{}F,\neg{}G,\boldsymbol{\neg{}H}\parallel
\neg{}A,A\lor\neg{}B,B\lor{}C,\\
~~~~B\lor\neg{}D,D\lor\neg{}E\lor\neg{}F,A\lor\neg{}G,G\lor{}E,G\lor\neg{}D\\
~~~~A\lor\neg{}H & \longrightarrow & (\mathrm{learn})\\

\neg{}A,\neg{}B,C,\neg{}D,E,\neg{}F,\neg{}G,\neg{}H\parallel
\neg{}A,A\lor\neg{}B,B\lor{}C,\\
~~~~B\lor\neg{}D,D\lor\neg{}E\lor\neg{}F,A\lor\neg{}G,G\lor{}E,G\lor\neg{}D\\
~~~~A\lor\neg{}H,\boldsymbol{H\lor{}F},\boldsymbol{H\lor{}\neg{}D} &
\longrightarrow & (\mathrm{unsat})\\

\mathrm{unsat}
\end{array}$
\end{center}
\begin{flushleft}
$\begin{array}{l}
\mbox{where:}\\
\begin{array}{ll}
~~~~A\equiv\lfloor\exists{}x.P(x)\Rightarrow{}P(a)\land{}P(b)\rfloor &
~~~~B\equiv\lfloor{}P(X)\Rightarrow{}P(a)\land{}P(b)\rfloor\\
~~~~C\equiv\lfloor{}P(X)\rfloor{} &
~~~~D\equiv\lfloor{}P(a)\land{}P(b)\rfloor{}\\
~~~~E\equiv\lfloor{}P(a)\rfloor &
~~~~F\equiv\lfloor{}P(b)\rfloor\\
~~~~G\equiv{}\lfloor{}P(a)\Rightarrow{}P(a)\land{}P(b)\rfloor &
~~~~H\equiv{}\lfloor{}P(b)\Rightarrow{}P(a)\land{}P(b)\rfloor
\end{array}
\end{array}$
\end{flushleft}}
\caption{Example of Proof Using the Tableau Theory}
\label{fig:tab}
\end{figure}

\subsection{The Rewriting Theory}
\label{sec:rew}

A rewriting theory allows us to introduce a computational behavior to the SAT
solver. We aim to integrate rewriting in the broadest sense of the term as
proposed by deduction modulo theory. Deduction modulo theory~\cite{DA03} focuses
on the computational part of a theory, where axioms are transformed into rewrite
rules, which induces a congruence over propositions, and where reasoning is
performed modulo this congruence. In deduction modulo theory, this congruence is
then induced by a set of rewrite rules over both terms and propositions.

In the following, we borrow some of the notations and definitions
of~\cite{DA03}. We call $\mathrm{FV}$ the function that returns the set of
free variables of a term or a formula. A term rewrite rule is a pair of terms
denoted by $l\rew{}r$, where $\mathrm{FV}(r)\subseteq\mathrm{FV}(l)$. A
proposition rewrite rule is a pair of propositions denoted by $l\rew{}r$, where
$l$ is an atomic proposition and $r$ is an arbitrary proposition, and where
$\mathrm{FV}(r)\subseteq\mathrm{FV}(l)$. A class rewrite system is a pair of
rewrite systems, denoted by $\mathcal{RE}$, consisting of $\mathcal{R}$, a set
of proposition rewrite rules, and $\mathcal{E}$, a set of term rewrite rules.

Given a class rewrite system $\mathcal{RE}$, the relations $=_\mathcal{E}$ and
$=_\mathcal{RE}$ are the congruences generated respectively by the sets
$\mathcal{E}$ and $\mathcal{R}\cup\mathcal{E}$. In the following, we use the
standard concepts of subterm and term replacement: given an occurrence $\omega$
in a proposition $P$, we write $P_{|\omega}$ for the term or proposition at
$\omega$, and $P[t]_\omega$ for the proposition obtained by replacing
$P_{|\omega}$ by $t$ in $P$ at $\omega$. Given a class rewrite system
$\mathcal{RE}$, the proposition $P$ $\mathcal{RE}$-rewrites to $P'$, denoted by
$P\rew_\mathcal{RE}P'$, if $P=_\mathcal{E}Q$, $Q_{|\omega}=\sigma(l)$, and
$P'=_\mathcal{E}Q[\sigma(r)]_\omega$, for some rule $l\rew{}r\in\mathcal{R}$,
some proposition $Q$, some occurrence $\omega$ in $Q$, and some substitution
$\sigma$.

The relation $=_\mathcal{RE}$ is not decidable in general, but there are some
cases where this relation is decidable depending on the class rewrite system
$\mathcal{RE}$ and the rewrite relation $\rew_\mathcal{RE}$. In particular, if
the rewrite relation $\rew_\mathcal{RE}$ is confluent and (weakly) terminating,
then the relation $=_\mathcal{RE}$ is decidable.

The rewriting theory is integrated into the SAT solver in a similar way as for
the tableau theory. Whenever a literal is propagated or decided, we generate a
clause introducing a new formula that is congruent to the formula in the box of
the literal. More precisely, given a literal $\lfloor{}P\rfloor$, where $P$ is a
formula, and a formula $P'$ such that $P=_\mathcal{RE}P'$, we generate the
following clause:

$$\left(\bigvee_{(l,r)\in\mathcal{R}}
\neg\lfloor\forall{}\vec{x}.l\Leftrightarrow{}r\rfloor\right)\lor
\left(\bigvee_{(l,r)\in\mathcal{E}}\neg\lfloor\forall{}\vec{x}.l=r\rfloor\right)\lor
\lfloor{}P\Leftrightarrow{}P'\rfloor$$

where $\vec{x}=\mathrm{FV}(l)\cup\mathrm{FV}(r)$.

It should be noted that in usual SMT solvers, rewriting can be emulated by means
of triggers that are actually the left-hand side members of the class rewrite
system $\mathcal{RE}$ introduced above. But in our rewriting theory, we can
generate the formula where rewriting setps have been done, while triggers can just
generate bindings, i.e. instances of the rewrite rules, which are used later to
relate the initial and rewritten formulas. Moreover, in our case, we can perform
several rewritings at once, while a trigger can only emulate one rewriting at a
time.

Let us illustrate the use of the rewriting theory by means of an example in set
theory. Let us prove that
$(\forall{}s,t.s\subseteq{}t\Leftrightarrow{}\forall{}x.x\in{}s\Rightarrow{}
x\in{}t)\Rightarrow{}a\subseteq{}a$, where $a$ is a constant. The proof is given
in Fig.~\ref{fig:rew}, where the rules ``$\mathrm{learn~tab}$'' and
``$\mathrm{learn~rew}$'' represent the rule ``$\mathrm{learn}$'' using
respectively the tableau and rewriting theories.

\begin{figure}[t!]
\parbox{\textwidth}
{\small
\begin{center}
$\begin{array}{lcl}
\emptyset\parallel
\neg{}A & \longrightarrow & (\mathrm{unit~prop})\\

\boldsymbol{\neg{}A}\parallel{}\neg{}A & \longrightarrow &
(\mathrm{learn~tab})\\

\neg{}A\parallel
\neg{}A,\boldsymbol{A\lor{}B},\boldsymbol{A\lor\neg{}C} & \longrightarrow &
(\mathrm{unit~prop})\times{}2\\

\neg{}A,\boldsymbol{B},\boldsymbol{\neg{}C}\parallel
\neg{}A,A\lor{}B,A\lor\neg{}C & \longrightarrow & (\mathrm{learn~rew})\\

\neg{}A,B,\neg{}C\parallel\neg{}A,A\lor{}B,A\lor\neg{}C,
\boldsymbol{\neg{}B\lor{}D} & \longrightarrow & (\mathrm{unit~prop})\\

\neg{}A,B,\neg{}C,\boldsymbol{D}\parallel\neg{}A,A\lor{}B,A\lor\neg{}C,
\neg{}B\lor{}D & \longrightarrow & (\mathrm{learn~tab})\\

\neg{}A,B,\neg{}C,D\parallel\neg{}A,A\lor{}B,A\lor\neg{}C,\neg{}B\lor{}D,
\boldsymbol{\neg{}D\lor{}E},\boldsymbol{\neg{}D\lor{}F} & \longrightarrow &
(\mathrm{unit~prop})\times{}2\\

\neg{}A,B,\neg{}C,D,\boldsymbol{E},\boldsymbol{F}\parallel
\neg{}A,A\lor{}B,A\lor\neg{}C,\neg{}B\lor{}D,\neg{}D\lor{}E,\\
~~~~\neg{}D\lor{}F & \longrightarrow & (\mathrm{learn~tab})\\

\neg{}A,B,\neg{}C,D,E,F\parallel\neg{}A,A\lor{}B,A\lor\neg{}C,\neg{}B\lor{}D,
\neg{}D\lor{}E\\
~~~~\neg{}D\lor{}F,\boldsymbol{\neg{}F\lor\neg{}G\lor{}C} & \longrightarrow &
(\mathrm{unit~prop})\\

\neg{}A,B,\neg{}C,D,E,F,\boldsymbol{\neg{}G}\parallel\neg{}A,A\lor{}B,
A\lor\neg{}C,\neg{}B\lor{}D,\neg{}D\lor{}E\\
~~~~\neg{}D\lor{}F,\neg{}F\lor\neg{}G\lor{}C & \longrightarrow &
(\mathrm{learn~tab})\\

\neg{}A,B,\neg{}C,D,E,F,\neg{}G\parallel\neg{}A,A\lor{}B,A\lor\neg{}C,
\neg{}B\lor{}D,\neg{}D\lor{}E\\
~~~~\neg{}D\lor{}F,\neg{}F\lor\neg{}G\lor{}C,\boldsymbol{G\lor\neg{}H} &
\longrightarrow & (\mathrm{unit~prop})\\

\neg{}A,B,\neg{}C,D,E,F,\neg{}G,\boldsymbol{\neg{}H}\parallel\neg{}A,A\lor{}B,
A\lor\neg{}C,\neg{}B\lor{}D\\
~~~~\neg{}D\lor{}E,\neg{}D\lor{}F,\neg{}F\lor\neg{}G\lor{}C,
G\lor\neg{}H & \longrightarrow & (\mathrm{learn~tab})\\

\neg{}A,B,\neg{}C,D,E,F,\neg{}G,\neg{}H\parallel\neg{}A,A\lor{}B,A\lor\neg{}C,
\neg{}B\lor{}D\\
~~~~\neg{}D\lor{}E,\neg{}D\lor{}F,\neg{}F\lor\neg{}G\lor{}C,
G\lor\neg{}H,\boldsymbol{H\lor{}I},\boldsymbol{H\lor\neg{}I} & \longrightarrow &
(\mathrm{unit~prop})\\

\neg{}A,B,\neg{}C,D,E,F,\neg{}G,\neg{}H,\boldsymbol{I}\parallel\neg{}A,A\lor{}B,
A\lor\neg{}C,\neg{}B\lor{}D\\
~~~~\neg{}D\lor{}E,\neg{}D\lor{}F,\neg{}F\lor\neg{}G\lor{}C,
G\lor\neg{}H,H\lor{}I,H\lor\neg{}I & \longrightarrow & (\mathrm{unsat})\\

\mathrm{unsat}
\end{array}$
\end{center}
\begin{flushleft}
$\begin{array}{l}
\mbox{where:}\\
\begin{array}{ll}
\multicolumn{2}{l}{
~~~~A\equiv\lfloor(\forall{}s,t.s\subseteq{}t\Leftrightarrow{}
\forall{}x.x\in{}s\Rightarrow{}x\in{}t)\Rightarrow{}a\subseteq{}a\rfloor}\\
~~~~B\equiv\lfloor\forall{}s,t.s\subseteq{}t\Leftrightarrow{}
\forall{}x.x\in{}s\Rightarrow{}x\in{}t\rfloor &
~~~~C\equiv\lfloor{}a\subseteq{}a\rfloor\\
~~~~D\equiv\lfloor{}a\subseteq{}a\Leftrightarrow
\forall{}x.x\in{}a\Rightarrow{}x\in{}a\rfloor &
~~~~E\equiv\lfloor{}a\subseteq{}a\Rightarrow
\forall{}x.x\in{}a\Rightarrow{}x\in{}a\rfloor\\
~~~~F\equiv\lfloor(\forall{}x.x\in{}a\Rightarrow{}x\in{}a)
\Rightarrow{}a\subseteq{}a\rfloor &
~~~~G\equiv\lfloor\forall{}x.x\in{}a\Rightarrow{}x\in{}a\rfloor\\
~~~~H\equiv\lfloor\epsilon_x\in{}a\Rightarrow\epsilon_x\in{}a\rfloor &
~~~~I\equiv\lfloor\epsilon_x\in{}a\rfloor
\end{array}\\\\
\mbox{with: }\epsilon_x=\epsilon(x).\neg(x\in{}a\Rightarrow{}x\in{}a)
\end{array}$
\end{flushleft}}
\caption{Example of Proof Using the Rewriting Theory}
\label{fig:rew}
\end{figure}

% $Id$

\section{Equational Reasoning with Rigid Unit Superposition}
\label{sec:super}

There are many ways of integrating equational reasoning in tableau
methods~\cite{DB75,LS02,BR15,DV96}. Because our prover does not rely on clausal
forms, but on arbitrary formulas with quantifiers occurring deep inside
branches, we deal with rigid variables, i.e. variables that can be instantiated
only once. In order to find instanciation as described in \ref{sec:tab},
we need to solve rigid E-unification modulo rewrite rules:~assume a set of
equations $E$, containing rigid variables, a rewrite system $\mathcal{RE}$,
and target terms $s$ and $t$; we want a substitution $\sigma$ such that
$\bigwedge_{e \in E} e\sigma \vdash s\sigma =_\mathcal{E} t\sigma$. Such a
substitution is a solution to the rigid E-unification problem.

We propose here an approach based on superposition with rigid variables, as in
previous work by Degtyarev and Voronkov~\cite{DV96} and earlier work on rigid
paramodulation~\cite{DAP00}, but with significant differences. First, in order
to avoid constraint solving, we do not use basic superposition nor
constraints. Second, we introduce a merging rule, which factors together
intermediate (dis)equations that are alpha-equivalent: with multiple instances
of some of the quantified formulas (amplification), it becomes important not to
duplicate work. In this aspect, our calculus is quite close to labeled unit
superposition~\cite{KS10} when using sets as labels. Third, unlike rigid
paramodulation, we use a term ordering to orient the equations.

\subsection{Preliminary Definitions}

We write $ \clauseWithSubst{ s \approx t }{ \Sigma}$ (resp.
$\clauseWithSubst{ s \not\approx t }{ \Sigma}$) for the unit clause that contains
exactly one equation (resp.~disequation) under hypothesis $\Sigma$ (which is a
set of substitutions). We write $\clauseWithSubst{\emptyset}{\Sigma}$ for the
empty clause under hypothesis $\Sigma$. We define $\renameVars{e}$, where $e$
is a (dis)equation, as follows: let $\sigma$ map every rigid variable of $e$ to
a fresh non-rigid variable, then
$\renameVars{e} = \clauseWithSubst{ e\sigma }{ \{ \sigma \} }$. For example,
$\renameVars{p(X)\approx a}$ is $\clauseWithSubst{ p(v1)\approx a}{ \{ X\mapsto
v1\} }$.
The meaning of $\clauseWithSubst{s \approx t}{ \Sigma}$ is
that for every $\sigma \in \Sigma$, $s \approx t$ is provable using the
substitution $\sigma$ for the metavariables.

As can be noticed, we keep a set of substitutions, rather than unit clauses
paired with individual substitutions, in order to avoid duplicating the work for
alpha-equivalent clauses. Indeed, because of amplification, many instances of a
given (dis)equation might be present in a branch of the tableau. It would be
inefficient to repeat the same inference steps with each variant of the axioms.
Because we apply $\renameVars{e}$ on every input equality $e$, clauses do not share any
variable, though they may share meta-variables in their attached sets of substitutions.

Considering a substitution as a function from variables to terms, we define the
domain of a substitution $\sigma$ as the set of variables that have a
non-trivial binding in $\sigma$.\footnote{A trivial binding maps a variable to
itself.} The co-domain of a substitution is the set of variables occurring in
terms in the image of the domain of the substitution. In the following, we will
consider idempotent substitutions, i.e. substitutions for which the domain and
co-domain have an empty intersection.

The composition of two substitutions $\sigma$ and $\sigma'$, denoted by
$\sigma \circ \sigma'$, is said to be well-defined if and only if the domains of
$\sigma$ and $\sigma'$ have no intersection. In this case,
$\sigma \circ \sigma' \triangleq \left\{ x \mapsto (x\sigma)\sigma' | x \in
\text{domain}(\sigma) \right\}$. This definition extends to sets of
substitutions: $\Sigma \circ \sigma' \triangleq \left\{ \sigma \circ \sigma' |
\sigma \in \Sigma \right\}$. We then have $\sigma \leq \sigma'$ if and only if
$\exists \sigma''.~ \sigma \circ \sigma'' = \sigma'$. This notion also extends
to sets of substitutions: $\smash{ \Sigma \leq \Sigma' }$ if and only if
$\smash{ \forall \sigma' \in \Sigma'.~ \exists \sigma \in \Sigma. \sigma \leq
\sigma' }$. The merging of two substitutions $\sigma \uparrow \sigma'$ is the
supremum of $\{\sigma,\sigma'\}$ for the order $\leq$, if it exists, or $\bot$
otherwise. The merging of sets of substitutions is
$\Sigma \uparrow \Sigma' \triangleq \left\{ \sigma \uparrow \sigma' ~|~ \sigma
\in \Sigma, \sigma' \in \Sigma' \right., \sigma \uparrow \sigma' \not= \bot
\}$.

To perform an inference step between two unit (dis)equations, we merge their
sets of substitutions. An inference rule is said to be successful if the merging
of the premises' substitution sets is non-empty. For example, the resolution step
between $p(x,x)| \{ X \mapsto a \}$ and $\lnot p(y,b)| \{ X \mapsto y \}$ is not
possible, because the result would need to map $X$ to $a$ and $b$, which is
impossible because $X$ is rigid.

\subsection{Inference System}

In Fig.~\ref{fig:unit-sup-rules}, we present the rules for unit superposition
with rigid variables. We adopt notations and names from Schulz's paper on
E~\cite{SS02}. A single bar denotes an inference, i.e. we add the result to the
saturation set, whereas a double bar is a simplification in which the premises
are replaced by the conclusion(s). The relation $\prec$ is a reduction ordering,
used to orient equations and restrict inferences, thus pruning the search space.
Typically, $\prec$ is one of RPO or KBO. The rules of
Fig.~\ref{fig:unit-sup-rules} work as described below:

\begin{description}
\item[ER] is equality resolution, where a disequation
$\clauseWithSubst{s \not\approx t}{\Sigma}$ is solved by syntactically unifying
$s$ and $t$ with $\sigma$, if $\sigma$ is compatible with $\Sigma$.
\item[SN/SP] is superposition into positive or negative literals. A subterm of $u$ is rewritten
using $s \approx t$ after unifying it with $s$ by $\sigma$. The rewriting is
done only if $s\sigma \not\preceq t\sigma$, a sufficient (but not necessary)
condition for a ground instance of $s\sigma \approx t\sigma$ to be oriented
left-to-right.
\item[TD1] deletes trivial equations that will never contribute to a proof.
\item[TD2] deletes clauses with an empty set of substitutions. In practice, we
only apply a rule if the conclusion is labeled with a non-empty set of
substitutions.
\item[ME] merges two alpha-equivalent clauses into a single clause, by merging
the sets of substitutions. This rule is very important in practice, to prevent
the search space from exploding due to the duplicates of most formulas.
Superposition deals with this explosion by removing duplicates using
subsumption, but in our context subsumption is not complete because rigid
variables are only proxy for ground terms: even if $C\sigma \subseteq D$, the
one ground instance of $C$ might not be compatible with the ground instance of
$D$.
\item[ES] is a restricted form of equality subsumption. The active equation
$\clauseWithSubst{ s\approx t}\Sigma $ can be used to delete another clause, as
in E~\cite{SS02}. However, ES only works if $s$ and $t$ are syntactically equal
to the corresponding subterms in the subsumed clause $C$. Otherwise, there is no
guarantee that further instantiations will not make $s\approx t$ incompatible
with $C$. Moreover, $C$ needs not be entirely removed. Only its substitutions
that are compatible with $\Sigma$ are subsumed.
\item[RN/RP] are rewriting of clauses, which only works for syntactical
equality, not matching.
\end{description}

Rule \textbf{SN/SP} generates as many equations as there are in the set
$(\Sigma \circ \sigma'') \uparrow (\Sigma' \circ \sigma'')$ because all
substitutions may not always be merged. For instance, given
$f(x) = t | \{ \{ X_1 \mapsto x \}, \{ X_2 \mapsto x \} \}$ and
$f(a) = v | \{ \{ X_1 \mapsto a \} \}$, we have to derive two distinct
non-mergeable equations $(t = v)\{ x \mapsto a \} | \{ \{ X_1 \mapsto a \} \}$
and $(t = v)\{ x \mapsto a \} | \{ \{ X_1 \mapsto a; X_2 \mapsto a \} \}$.

\begin{figure}[htb]
\begin{center}

% SN
\AXC{$\clauseWithSubst{s \approx t}{\Sigma}$}
\AXC{$\clauseWithSubst{u \mathrel{R} v}{\Sigma'}$}
\LL{SN/SP}
\BIC{$\clauseWithSubst{\sigma''(u[p \leftarrow t] \mathrel{R} v)}{\sigma'''}$}\DP
$\text{if} \left\{ \begin{array}{l@{\quad}l}
\sigma'' = \text{mgu}(u_{|p}, s) & u_{|p} \not\in V\\
\sigma''(s) \not\preceq \sigma''(t) & \sigma''(u) \not\preceq \sigma''(v)\\
\multicolumn{2}{l}{
\sigma''' \in (\Sigma \circ \sigma'') \uparrow (\Sigma' \circ \sigma'')} \\
\multicolumn{2}{l}{R \in \{ \approx, \not\approx \} }
\end{array}\right.$\\[12pt]

\mbox{
% ER
\AXC{$\clauseWithSubst{s \not\approx t}{\Sigma}$}
\LL{ER}
\RL{if $\sigma = \text{mgu}(s, t)$}
\UIC{$\clauseWithSubst{\emptyset}{\Sigma \circ \sigma}$}\DP

% TD1
\AXC{$\clauseWithSubst{s \approx s}{\Sigma}$}
\LL{TD1}
\doubleLine{}
\UIC{$\top$}\DP

% TD2
\AXC{$\clauseWithSubst{s \mathrel{R} t}{\emptyset}$}
\LL{TD2}
\RL{$ R \in \{ \approx, \not\approx \} $}
\doubleLine{}
\UIC{$\top$}
\DP}\\[12pt]

% ME
\AXC{$\clauseWithSubst{\rho(u) \approx \rho(v)}{\Sigma}$}
\AXC{$\clauseWithSubst{u \approx v}{\Sigma'}$}
\LL{ME}
\RL{$\rho \text{ is a variable renaming}$}
\doubleLine{}
\BIC{$\clauseWithSubst{\rho(u) \approx \rho(v)}{\Sigma \cup (\Sigma' \circ \rho)}$}\DP\\[12pt]

% ES
\AXC{$\clauseWithSubst{s \approx t}{\Sigma}$}
\AXC{$\clauseWithSubst{u[p \leftarrow s] \approx u[p \leftarrow t]}{\Sigma' \cup \Sigma''}$}
\LL{ES}
\RL{$
\text{if} \left\{ \begin{array}{l}
\Sigma'' \not= \emptyset\\
\Sigma \leq \Sigma''
\end{array}\right.$}
\doubleLine{}
\BIC{$\clauseWithSubst{s\approx t}{\Sigma} \qquad
      \clauseWithSubst{u[p\leftarrow s] \approx u[p \leftarrow t]}{\Sigma'}$}\DP\\[12pt]

% RP
\AXC{$\clauseWithSubst{s \approx t}{\Sigma}$}
\AXC{$\clauseWithSubst{u \approx v}{\Sigma'}$}
\LL{RP}
\doubleLine{}
\BIC{$\clauseWithSubst{s \approx t}{\Sigma}$ \qquad
     $\clauseWithSubst{u[p \leftarrow t] \approx v}{\Sigma'}$}\DP
$\text{if} \left\{\begin{array}{l}
u_{|p} = s\\
s \succ t\\
\Sigma \leq \Sigma'\\
u \not\succeq v ~ \text{or} ~ p \neq \lambda\\
\end{array}\right.$\\[12pt]

% RN
\AXC{$\clauseWithSubst{s \approx t}{\Sigma}$}
\AXC{$\clauseWithSubst{u \not\approx v}{\Sigma'}$}
\LL{RN}
\doubleLine{}
\BIC{$\clauseWithSubst{s \approx t}{\Sigma}$ \qquad
     $\clauseWithSubst{u[p \leftarrow t] \not\approx v}{\Sigma'}$}\DP
$\text{if} \left\{\begin{array}{l}
u_{|p} = s\\
s \succ t\\
\Sigma \leq \Sigma'\\
\end{array}\right.$
\caption{The Set of Rules for Unit Rigid Superposition}
\label{fig:unit-sup-rules}
\end{center}
\end{figure}

\subsection{Rewriting}

Rewrite rules can be integrated to the rigid unit superposition easily. In fact,
a rewrite rule $l\rew{}r$ can be expressed as an equality with a hypothesis set
consisting of a single trivial substitution
$\clauseWithSubst{s\approx{}t}{\{\emptyset\}}$\footnote{$\{\emptyset\}$ is not to be confused with $\emptyset$}.
Since the trivial substitution is compatible
with every substitution, it will never prevent any inference, thus allowing us
to use the unit clause as many times as needed to rewrite terms without
accumulating constraints, particularly using the rules RP and RN, whose side
conditions are always verified by rewrite rules. Rigid unit superposition
therefore provides an algorithm for rigid E-unification modulo rewrite rules,
as detailed in the next paragraph.

\subsection{Main Loop}

Our objective with rigid E-unification is to attempt to close a branch of the
tableau prover (i.e. a set of Boolean literals set to true). To do so, we first
create a set of unit clauses to process from the rewrite rules, and the renamed
equational or atomic literals. Then, the given-clause algorithm is applied to try and
saturate the set. Assuming a fair strategy, this will eventually find a
solution (i.e. derive $\clauseWithSubst{\emptyset}{\Sigma}$) if there exists
one. We refer the interested reader to~\cite{SS02} for more details.

Because the whole branch is managed by a single given-clause saturation loop, we
look for all solutions susceptible to close the branch at the same time.
Moreover, this technique is amenable to incrementality, i.e. every time a
(dis)equation is decided by the SAT solver, we could add it to the saturation
set and perform a (limited) number of steps of the given-clause algorithm.

\subsection{Example}

To illustrate the calculus, we detail a refutation of the following set of
clauses stemming from set theory, where pair, fst, and snd are the constructor
and destructors of tuples, $f$ a function on tuples, and $X$ a rigid variable:

\[\begin{array}{rcl}
a &\preceq b \\
\text{pair}(\text{fst}(x), \text{snd}(x))) &\rew& x\\
\text{fst}(a) &\approx& \text{fst}(b)\\
p(a) &\not\approx& p(\text{pair}(\text{fst}(b), X))\\
\end{array}\]

Because the problem is purely equational, the tableau structure is trivial, and
all the work is done by the rigid superposition procedure as shown in
Fig.~\ref{fig:unit-sup-proof-example}.

\begin{figure}[t]
\begin{center}
\begin{tabular}{clc}
1 & rewrite rule & $\text{pair}(\text{fst}(x), \text{snd}(x))) \rew x$\\

2 & axiom & $\text{fst}(a) = \text{fst}(b)$\\

3 & axiom & $p(a) \not= p(\text{pair}(\text{fst}(b), X))$\\

4 & \textsf{rewr}(1) &
$\clauseWithSubst
{ \text{pair}(\text{fst}(x), \text{snd}(x)) \approx x }
{ \{ \} }$\\

5 & \renameVarsSymb(2) &
$\clauseWithSubst
{ \text{fst}(a) \approx \text{fst}(b) }
{ \{ \} }$\\

6 & \renameVarsSymb(3) &
$\clauseWithSubst
{ f(a) \not\approx f(\text{pair}(\text{fst}(b), y)) }
{ \{ \mapVar{X}y \} }$\\

\midrule

7 & RN(5,6) &
$\clauseWithSubst
{ f(a) \not\approx f(\text{pair}(\text{fst}(a), y)) }
{ \{ \mapVar{X}y \} }$\\

8 & SN(4,7) &
$\clauseWithSubst
{ f(a) \not\approx f(a) }
{ \{ \mapVar{X}{\text{snd}(a)} \} }$\\

9 & ER(8) &
$\clauseWithSubst
{ \emptyset }
{ \{ \mapVar{X}{\text{snd}(a)} \} }$
\end{tabular}
\caption{Proof of a Set Theory Problem}
\label{fig:unit-sup-proof-example}
\end{center}
\end{figure}

% $Id$

\section{Implementation and Experimental Results}
\label{sec:bench}

In this section, we briefly describe the implementation of our approach
introduced previously, and present some experimental results obtained by running
this implementation over a benchmark of problems in the \bmth{} set theory.

The algorithms described in this paper are implemented in the \archsat{}
automated theorem prover\footnote{Available at:
\url{https://gforge.inria.fr/projects/archsat}.}. It relies on the
\msat{}~\cite{GB17} library, derived from the \altergoz{} tool, which is a
generic library for building automated deduction tools based on SAT solvers.
\archsat{} (as well as \msat{}) is written in \ocaml{}. \archsat{} natively
supports polymorphic types as described in~\cite{BP13}.

\subsection{Experimental Results}

As a framework to test our tool, we consider the set theory of the \bmth{}
method~\cite{B-Book}. This method is supported by some tool sets, such as
\atelierb{}, which are used in industry to specify and build, by stepwise
refinements, software that is correct by design. This theory is suitable as it
can be easily turned into a theory that is compatible with deduction modulo
theory, i.e. where a large part of axioms can be turned into rewrite rules, and
for which the rewriting theory proposed previously in Subsec.~\ref{sec:rew}
works. Starting from the theory described in Chap.~2 of the
\bbook{}~\cite{B-Book}, we therefore transform whenever possible the axioms and
definitions into rewrite rules. The resulting theory has been introduced
in~\cite{BA15}. As can be seen, the proposed theory is typed, using first order logic extended
to polymorphic types à la ML, through a type system in the spirit
of~\cite{BP13}. This extension to polymorphic types offers more flexibility, and
in particular allows us to deal with theories that rely on elaborate type
systems, like the \bmth{} set theory (see Chap.~2 of the
\bbook{}~\cite{B-Book}).

To test \archsat{} in this theory, we consider 319~lemmas coming from Chap.~2 of
the \bbook{}~\cite{B-Book}\footnote{Available at: \url{https://github.com/delahayd/bset}}.
These lemmas are properties of various difficulty
regarding the set constructs introduced by the \bmth{} method. It should be
noted that these constructs and notations are, for a large part of them,
specific to the \bmth{} method, as they are used for the modeling of industrial
projects, and are not necessarily standard in set theory.

As tools, we consider \archsat{} (development version\footnote{\git{}
branch smt18.}). We also include other automated theorem provers, able to
deal with first order logic with polymorphic types and rewriting natively. In
particular, we consider \zenm{} (version~0.4.2), a tableau-based prover that
is an extension of \zenon{} to deduction modulo theory. To show the impact of
rewriting on the results, we also include the \altergo{} SMT solver
(version~1.01). It would have been possible to also consider provers dealing
with pure first order logic and encode the polymorphic layer. But preliminary
tests have been conducted and very low results have been obtained even for the
best state-of-the-art provers (we have considered \e{} and \cvc{} in
particular), which indicates that polymorphism encoding adds a lot of noise
in proof search and is not effective in practice.

The experiment was run on an \intel{}~3.50~GHz computer, with a timeout of 90s
(beyond this timeout, results do not change) and a memory limit of 1~GiB. The
results are summarized in Tab.~\ref{tab:bench}. In these results, we observe
that \archsat{} obtains better results, in terms of proved problems, than
\zenm{} and \altergo{}, which tends to show the effectiveness of our approach in
practice. Looking at the cumulative times, \altergo{} is not really faster than
\archsat{}, which take more time to find few more difficult
problems: with a timeout of 3s, \archsat{} finds 260 proofs in 16.61~s,
while \altergo{} obtains the same results.

\setlength{\tabcolsep}{3pt}
\renewcommand{\arraystretch}{1.2}
\newcolumntype{C}{>{\centering}X}

\begin{table}[t]
\begin{center}
\begin{tabularx}{\textwidth}{|X|C|C|C|}
\hline
\begin{tabular}{l}
319~Problems
\end{tabular} &
\begin{tabular}{c}
\archsat{}
\end{tabular} &
\begin{tabular}{c}
\zenm{}
\end{tabular} &
\begin{tabular}{c}
\altergo
\end{tabular}\tabularnewline
\hline
\begin{tabular}{l}
Proofs
\end{tabular} &
\begin{tabular}{c}
272
\end{tabular} &
\begin{tabular}{c}
138
\end{tabular} &
\begin{tabular}{c}
232
\end{tabular}\tabularnewline
\hline
\begin{tabular}{l}
Rate
\end{tabular} &
\begin{tabular}{c}
85.3\%
\end{tabular} &
\begin{tabular}{c}
43.3\%
\end{tabular} &
\begin{tabular}{c}
72.7\%
\end{tabular}\tabularnewline
\hline
\begin{tabular}
Total time \small{(s)}
\end{tabular} &
\begin{tabular}{c}
268.69
\end{tabular} &
\begin{tabular}{c}
2.86
\end{tabular} &
\begin{tabular}{c}
8.42
\end{tabular}\tabularnewline
\hline
\end{tabularx}
\end{center}
\caption{Experimental Results over the \bmth{} Set Theory Benchmark}
\label{tab:bench}
\end{table}

\renewcommand{\arraystretch}{1}

% $Id$

\section{Conclusion}

We have described the architecture of \archsat{}, an automated theorem prover
that combines a SAT solver with tableau calculus and rewriting. Compared to
several other tools, \archsat{} appears quite effective in practice, as shown by
some experimental results obtained by running our implementation over a
benchmark of problems in the \bmth{} set theory.

As perspectives, we plan to realize more tests of \archsat{} over other theories
where a large part of these theories can be turned into rewrite rules. In
particular, a regular trigger mechanism has been also implemented in \archsat{}
and can be used to deal with conditional rewriting (the instantiation is delayed
and performed once the condition has been evaluated to true). This feature
should open up a range of new perspectives on the theories that our approach
could handle. We also aim to apply our tool to the benchmark of the \bware{}
project~\cite{BWare}, which consists of a large collection of proof obligations
coming from the development of industrial applications using the \bmth{} method.
This collection gathers about 13,000~problems, and should allow us to understand
to what extent our tool scales up. To do so, we have to extend \archsat{} to
arithmetic, which can be done using the modular architecture of \archsat{} and
adding an arithmetic theory to the SAT solver core of the prover.


\bibliographystyle{abbrv}
\bibliography{biblio}

\end{document}
