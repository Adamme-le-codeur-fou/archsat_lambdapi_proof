% $Id$

\section{Equational Reasoning with Rigid Unit Superposition}
\label{sec:super}

There are many ways of integrating equational reasoning in Tableaux
methods~\cite{DB75,LS02,BR15,DV96}. Because our prover does not rely on clausal
forms, but on arbitrary formulas with quantifiers occurring deep inside
branches, we deal with {\em rigid} variables, i.e. variables that can be
instantiated only once. The problem we want to solve, {\em rigid E-unification
modulo rewrite rules}, is the following. Assume a set of equations $E$,
containing rigid variables, a rewrite system $\mathcal{RE}$, and {\em target
terms} $s$ and $t$. We want a substitution $\sigma$ such that
$\bigwedge_{e \in E} e\sigma \vdash s\sigma =_\mathcal{E} t\sigma$; such a
substitution is a {\em solution} to the rigid E-unification problem.

We propose here an approach based on superposition with rigid variables, as in
previous work by Degtyarev and Voronkov~\cite{DV96} and earlier work on rigid
paramodulation~\cite{DAP00}, but with significant differences. First, to avoid
constraint solving, we do not use basic superposition nor constraints. Second,
we introduce a {\em merging} rule that factors together intermediate
(dis)equations that are alpha-equivalent: with multiple instances of some of the
quantified formulas ({\em amplification}), it becomes important not to duplicate
work. In this aspect, our calculus is quite close to labeled unit
superposition~\cite{KS10} when using sets as labels. Third, unlike rigid
paramodulation we use a term ordering to orient the equations.

%Mention whether other rigid E-unification modulo rewrite algorithm exists.

\subsection{Preliminary Definitions}

We write $ \clauseWithSubst{ s \approx t }{ \Sigma}$ (resp.
$\clauseWithSubst{ s \not\approx t }{ \Sigma}$), the unit clause that contains
exactly one equation (resp.~disequation) under hypothesis $\Sigma$ (which is a
set of substitutions). We write $\clauseWithSubst{\emptyset}{\Sigma}$ for the
empty clause under hypothesis $\Sigma$. We define $\renameVars{e}$, where $e$
is a (dis)equation, as follows: let $\sigma$ map every rigid variable of $e$ to
a fresh non-rigid variable; then
$\renameVars{e} = \clauseWithSubst{ e\sigma }{ \{ \sigma \} }$. For example,
$\renameVars{p(X)\approx a}$ is $\clauseWithSubst{ p(Y)\approx a}{ \{ X\mapsto
Y\} }$. The E-unification problem $E \vdash s\approx t$ can be solved by proving
$\clauseWithSubst{\emptyset }{ \Sigma}$ from
$\{ \renameVars{e} \}_{ e \in E } \cup \{ \renameVars{ s \not\approx t } \}$,
where $\Sigma$ contains the solutions. The meaning of $s \approx t | \Sigma$ is
that for every $\sigma \in \Sigma$, $s \approx t$ is provable using the
substitution $\sigma$ for the metavariables.

As can be noticed, we keep a set of substitutions, rather than unit clauses
paired with individual substitutions, in order to avoid duplicating the work for
alpha-equivalent clauses. Indeed, because of amplification, many instances of a
given (dis)equation might be present in a branch of the tableau; it would be
inefficient to repeat the same inference steps with each variant of the axioms.
Because we apply $\renameVars{e}$ on every initial $e$, clauses do not share any
variable, except in their attached sets of substitutions.

To perform an inference step between two unit (dis)equations, we merge their
sets of substitutions. Merging $\Sigma$ and $\Sigma'$, intuitively, means
computing $\{ \textsf{merge}(\sigma,\sigma') ~|~ \sigma \in \Sigma, \sigma'\in
\Sigma' \}$ for every pair $(\sigma,\sigma')$ of {\em compatible} substitutions.
For example, the resolution step between $p(x,x)| \{ X \mapsto a \}$ and $\lnot
p(y,b)| \{ X \mapsto y \}$ is not possible, because the result would need to map
$X$ to $a$ and to $b$, which is impossible because $X$ is rigid. Compatibility
relies on a partial ordering $\leq$, such that $\sigma \leq \sigma'$ means that
$\sigma$ is less general than $\sigma'$.

Considering a substitution as a function from variables to terms, we can define
the domain of a substitution $\sigma$ as the set of variables which have a
non-trivial binding in $\sigma$.\footnote{A trivial binding maps a variable to
itself.} The co-domain of a substitution is the set of variables occurring in
terms in the image of the domain of the substitution. In all the following, we
will consider idempotent substitutions, i.e. substitution for which the domain
and co-domain have an empty intersection.

The {\em composition} of substitutions $\sigma \circ \sigma'$ is well-defined
iff the domains of $\sigma$ and $\sigma'$ have no intersection. In this case,
$\sigma \circ \sigma' \triangleq \left\{ x \mapsto (x\sigma)\sigma' | x \in
\text{domain}(\sigma) \right\}$. This definition extends to sets of
substitutions: $\Sigma \circ \sigma' \triangleq \left\{ \sigma \circ \sigma' |
\sigma \in \Sigma \right\}$. We then have $\sigma \leq \sigma'$ iff $\exists
\sigma''.~ \sigma \circ \sigma'' = \sigma'$. This notion also extends to sets of
substitutions: $\smash{ \Sigma \leq \Sigma' }$ iff $\smash{ \forall \sigma' \in
\Sigma'.~ \exists \sigma \in \Sigma. \sigma \leq \sigma' }$. The {\em merging}
of two substitutions $\sigma \uparrow \sigma'$ is the supremum of
$\{\sigma,\sigma'\}$ for the order $\leq$, if it exists, or $\bot$ otherwise.
The merging of sets of substitutions is $\Sigma \uparrow \Sigma' \triangleq
\left\{ \sigma \uparrow \sigma' ~|~ \sigma \in \Sigma, \sigma' \in \Sigma'
\right., \sigma \uparrow \sigma' \not= \bot \}$. An inference rule is said to be
successful if the merging of the premises' substitution sets is non-empty.

\subsection{Inference System}

In Fig.~\ref{fig:unit-sup-rules}, we present the rules for unit superposition
with rigid variables. We adopt notations and names from Schulz's paper on
E~\cite{SS02}. A single bar denotes an inference, i.e. we add the result to the
saturation set,whereas a double bar is a simplification in which the premises
are replaced by the conclusion(s). The relation $\prec$ is a {\em reduction
ordering}, used to orient equations and restrict inferences, thus pruning the search space.
Typically,
$\prec$ is one of RPO or KBO. The rules of Fig.~\ref{fig:unit-sup-rules} work as
described below:

\begin{description}
\item[ER] is {\em equality resolution}, where a disequation
$\clauseWithSubst{s \not\approx t}\Sigma$ is solved by syntactically unifying
$s$ and $t$ with $\sigma$, if $\sigma$ is compatible with $\Sigma$.
\item[SN] is superposition into negative literals. A subterm of $u$ is rewritten
using $s \approx t$ after unifying it with $s$ by $\sigma$. The rewriting is
done only if $s\sigma \not\preceq t\sigma$, a sufficient (but not necessary)
condition for a ground instance of $s\sigma \approx t\sigma$ to be oriented
left-to-right.
\item[SP] is similar to SN, but superposes into a positive literal.
\item[TD1] deletes trivial equations that will never contribute to a proof.
\item[TD2] deletes clauses with an empty set of substitutions. In practice, we
only apply a rule if the conclusion is labeled with a non-empty set of
substitutions.
\item[ME] merges two alpha-equivalent clauses into a single clause, by merging
the sets of substitutions. This rule is very important in practice, to prevent
the search space from exploding due to the duplicates of most formulas.
Superposition deals with this explosion by removing duplicates using
{\em subsumption}, but in our context subsumption is not complete because rigid
variables are only proxy for ground terms: even if $C\sigma \subseteq D$, the
one ground instance of $C$ might not be compatible with the ground instance of
$D$.
\item[ES] is a restricted form of equality subsumption. The active equation
$\clauseWithSubst{ s\approx t}\Sigma $ can be used to delete another clause, as
in E~\cite{SS02}. However, ES only works if $s$ and $t$ are syntactically equal
to the corresponding subterms in the subsumed clause $C$; otherwise, there is no
guarantee that further instantiations will not make $s\approx t$ incompatible
with $C$. Moreover, $C$ needs not be entirely removed; only its substitutions
that are compatible with $\Sigma$ are actually subsumed.
\item[RP] similarly, rewriting of positive clauses only works for syntactical
equality, not matching.
\item[RN] is the same as RP but for rewriting negative clauses.
\end{description}

\begin{figure}[htb]
\begin{center}
% ER
\AXC{$s \not\approx t |\Sigma$}
\LL{ER}
\RL{if $\sigma = \text{mgu}(s, t)$}
\UIC{$\emptyset | \Sigma \circ \sigma $}\DP\\[12pt]

% SN
\AXC{$s \approx t | \Sigma$}
\AXC{$u \not\approx v | \Sigma'$}
\LL{SN}
\BIC{$\sigma''(u[p \leftarrow t] \not\approx v) | (\Sigma \circ \sigma'')
\uparrow (\Sigma' \circ \sigma'')$}\DP
$\text{if} \left\{ \begin{array}{l}
\sigma'' = \text{mgu}(u_{|p}, s)\\
\sigma''(s) \not\preceq \sigma''(t)\\
\sigma''(u) \not\preceq \sigma''(v)\\
u_{|p} \not\in V\\
\end{array}\right.$\\[12pt]

% SP
\AXC{$s \approx t | \Sigma$}
\AXC{$u \approx v | \Sigma'$}
\LL{SP}
\BIC{$\sigma''(u[p \leftarrow t] \approx v) | (\Sigma \circ \sigma'') \uparrow
(\Sigma' \circ \sigma'')$}\DP
$\text{if} \left\{ \begin{array}{l}
\sigma'' = \text{mgu}(u_{|p}, s)\\
\sigma''(s) \not\preceq \sigma''(t)\\
\sigma''(u) \not\preceq \sigma''(v)\\
u_{|p} \not\in V\\
\end{array}\right.$\\[12pt]

\mbox{
% TD1
\AXC{$s \approx s | \Sigma $}
\LL{TD1}
\doubleLine{}
\UIC{$\top$}\DP

% TD2
\AXC{$s \mathrel{R} t | \emptyset$}
\LL{TD2}
\RL{$ R \in \{ \approx, \not\approx \} $}
\doubleLine{}
\UIC{$\top$}
\DP}\\[12pt]

% PS
%\AXC{$s \approx t |\sigma$}
%\AXC{$u[p \leftarrow \sigma''(s)] \not\approx u[p \leftarrow \sigma''(t)] |
%\sigma' $}
%\LL{PS}
%\doubleLine{}
%\BIC{$s \approx t | \sigma$ \qquad $\emptyset | \sigma \cup \sigma' \cup
%\sigma''$}
%\DP{} \\[12pt]

% NS
%\AXC{$s \not\approx t | \sigma$}
%\AXC{$\sigma''(s \approx t) | \sigma'$}
%\LL{NS}
%\doubleLine{}
%\BIC{$s \not\approx t | \sigma$ \qquad $\emptyset | \sigma \cup \sigma' \cup
%\sigma''$}
%\DP{} \\[12pt]

% ME
\AXC{$\rho(u) \approx \rho(v) | \Sigma$}
\AXC{$u \approx v | \Sigma'$}
\LL{ME}
\RL{$\rho \text{ is a variable renaming}$}
\doubleLine{}
\BIC{$\rho(u) \approx \rho(v) | \Sigma \cup (\Sigma' \circ \rho)$}\DP\\[12pt]

% ES
\AXC{$s \approx t | \Sigma$}
\AXC{$u[p \leftarrow s] \approx u[p \leftarrow t] | \Sigma' \cup \Sigma''$}
\LL{ES}
\RL{$
\text{if} \left\{ \begin{array}{l}
\Sigma'' \not= \emptyset\\
\Sigma \leq \Sigma''
\end{array}\right.$}
\doubleLine{}
\BIC{$s\approx t | \Sigma \qquad u[p\leftarrow s] \approx u[p \leftarrow t] |
\Sigma'$}\DP\\[12pt]

% RP
\AXC{$s \approx t | \Sigma$}
\AXC{$u \approx v | \Sigma'$}
\LL{RP}
\doubleLine{}
\BIC{$s \approx t | \Sigma$ \qquad $u[p \leftarrow t] \approx v | \Sigma'$}\DP
$\text{if} \left\{\begin{array}{l}
u_{|p} = s\\
s \succ t\\
\Sigma \leq \Sigma'\\
u \not\succeq v ~ \text{or} ~ p \neq \lambda\\
\end{array}\right.$\\[12pt]

% RN
\AXC{$s \approx t | \emptyset$}
\AXC{$u \not\approx v | \sigma$}
\LL{RN}
\doubleLine{}
\BIC{$s \approx t | \emptyset$ \qquad $u[p \leftarrow t] \not\approx |
\sigma$}\DP
$\text{if} \left\{\begin{array}{l}
u_{|p} = s\\
s \succ t\\
\Sigma \leq \Sigma'\\
\end{array}\right.$
\caption{The Set of Rules for Unit Rigid Superposition}
\label{fig:unit-sup-rules}
\end{center}
\end{figure}

\subsection{Rewriting}

Rewrite rules can be integrated to the rigid unit superposition easily. In fact,
a rewrite rule $l\rew{}r$ can be expressed as an equality with a hypothesis set
consisting of a single trivial substitution
$s\approx{}t|\{\text{identity}\}$. Since the trivial substitution is compatible
with every substitution, it will never prevent any inference, thus allowing to
use the unit clause as many times as needed to rewrite terms without
accumulating constraints, particularly using the rules RP and RN, whose side
conditions are always verified by rewrite rules. Rigid unit superposition thus
provides an algorithm for rigid E-unification modulo rewrite rules.

\subsection{Main Loop}

Our objective with rigid E-unification is to attempt to close a branch of the
tableau prover (i.e., a set of Boolean literals set to true). To do so, all
equational or atomic literals are added to a set of unit clauses to process,
with a label $\Sigma \triangleq \{ \emptyset \}$. Then, the given-clause
algorithm is applied to try and saturate the set. Assuming a fair strategy, this
will eventually find a solution (i.e. derive
$\clauseWithSubst{\emptyset}{\Sigma}$) if there exists one. We refer the
interested reader to~\cite{SS02} for more details.

Because the whole branch is treated by a single given-clause saturation loop, we
look for all solutions susceptible to close the branch at the same time.
Moreover, this technique is amenable to incrementality, i.e. every time a
(dis)equation is decided by the SAT solver, we could add it to the saturation
set and perform a (limited) number of steps of the given-clause algorithm.

\subsection{Example}

To illustrate the calculus, we detail a refutation of the following set of
clauses stemming from set theory, where pair, fst, and snd are the constructor
and destructors of tuples, $f$ a function on tuples, and $X$ a rigid variable:

\[\begin{array}{rcl}
\text{pair}(\text{fst}(x), \text{snd}(x))) &\rew& x\\
\text{fst}(a) &\approx& \text{fst}(b)\\
p(a) &\not\approx& p(\text{pair}(\text{fst}(b), X))\\
\end{array}\]

Because the problem is purely equational, the tableau structure is trivial, and
all the work is done by the rigid superposition procedure as shown in
Fig.~\ref{fig:unit-sup-proof-example}.

\begin{figure}[t]
\begin{center}
\begin{tabular}{clc}
1 & axiom & $\text{pair}(\text{fst}(x), \text{snd}(x))) \rew x$\\

2 & axiom & $\text{fst}(a) = \text{fst}(b)$\\

3 & axiom & $p(a) \not= p(\text{pair}(\text{fst}(b), X))$\\

4 & \renameVarsSymb(1) &
$\clauseWithSubst
{ \text{pair}(\text{fst}(x), \text{snd}(x) \approx x }
{ \{ \} }$\\

5 & \renameVarsSymb(2) &
$\clauseWithSubst
{ \text{fst}(a) \approx \text{fst}(b) }
{ \{ \} }$\\

6 & \renameVarsSymb(3) &
$\clauseWithSubst
{ f(a) \not\approx f(\text{pair}(\text{fst}(b), y)) }
{ \{ \mapVar{X}y \} }$\\

\midrule

7 & RN(5,6) &
$\clauseWithSubst
{ f(a) \not\approx f(\text{pair}(\text{fst}(a), y)) }
{ \{ \mapVar{X}y \} }$\\

8 & SN(4,7) &
$\clauseWithSubst
{ f(a) \not\approx f(a) }
{ \{ \mapVar{X}{\text{snd}(a)} \} }$\\

9 & ER(8) &
$\clauseWithSubst
{ \emptyset }
{ \{ \mapVar{X}{\text{snd}(a)} \} }$
\end{tabular}
\caption{Proof of a Set Theory Problem}
\label{fig:unit-sup-proof-example}
\end{center}
\end{figure}
