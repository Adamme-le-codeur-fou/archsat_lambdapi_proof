% $Id$

\section{Implementation and Experimental Results}
\label{sec:bench}

\subsection{Implementation}

%The implementation relies on the
%\msat{} library, which derives from \altergoz{}, and which is a generic library
%that allows us to easily create automated deduction tools based on SAT solvers.

\subsection{Experimental Results}

As a framework to test our tool, we consider the set theory of the \bmth{}
method~\cite{B-Book}. This method is supported by some tool sets, such as
\atelierb{}, which are used in industry to specify and build, by stepwise
refinements, software that is correct by design. This theory is suitable as it
can be easily turned into a theory that is compatible with deduction modulo
theory, i.e. where a large part of axioms can be turned into rewrite rules, and
for which the rewriting theory proposed previously in Subsec.~\ref{sec:rew}
should work. Starting from the theory described in Chap.~2 of the
\bbook{}~\cite{B-Book}, we therefore transform whenever possible the axioms and
definitions into rewrite rules. The resulting theory has been introduced
in~\cite{BA15}, and due to lack of space, we only provide, in
Fig.~\ref{fig:bset}, the three rewriting rules corresponding to the axiomatic
core of the \bmth{} set theory that we consider.

As can be seen, the proposed theory is typed, using first order logic extended
to polymorphic types à la ML, through a type system in the spirit
of~\cite{BP13}. This extension to polymorphic types offers more flexibility, and
in particular allows us to deal with theories that rely on elaborate type
systems, like the \bmth{} set theory (see Chap.~2 of the
\bbook{}~\cite{B-Book}). The complete type system that is used in this
formalization can be found in~\cite{BA15}. The type constructors,
i.e. $\mathsf{tup}$ for tuples and $\mathsf{set}$ for sets, and type schemes of
the considered set constructs are provided in Fig.~\ref{fig:bset} as well, where
\type{} is the type of types and $\omicron$ the type of formulas, and where type
arguments are subscript annotations of the constructs.

\begin{figure}[t]
\small
\hspace{0.2cm}\underline{Axioms of Set Theory}
\begin{flushleft}
$\begin{array}{@{\hspace{0.2cm}}l}
(x,y)_{\alpha_1,\alpha_2}\in_{\tuple{\alpha_1}{\alpha_2}}s\times_{\alpha_1,\alpha_2}t\rew
x\in_{\alpha_1}s\land{}y\in_{\alpha_2}t\\
s\in_{\set{\alpha}}\mathbb{P}_\alpha(t)\rew
\forall{}x:\alpha.x\in_\alpha{}s\Rightarrow{}x\in_\alpha{}t\\
s=_{\set{\alpha}}t\rew
\forall{}x:\alpha.x\in_\alpha{}s\Leftrightarrow{}x\in_\alpha{}t
\end{array}$
\end{flushleft}
\hspace{0.2cm}\underline{Type Constructors}
\begin{flushleft}
$\begin{array}{@{\hspace{0.2cm}}l@{\hspace{1.0cm}}l}
\mathsf{tup}:\Pi\alpha_1,\alpha_2:\type.\type &
\mathsf{set}:\Pi\alpha:\type.\type
\end{array}$
\end{flushleft}
\hspace{0.2cm}\underline{Type Schemes of the Set Constructs}
\begin{flushleft}
$\begin{array}{@{\hspace{0.2cm}}lcl}
\arg\in\arg & : &
\Pi\alpha:\type.\alpha\rightarrow\set{\alpha}\rightarrow\omicron\\
(\arg,\arg) & : &
\Pi\alpha_1,\alpha_2:\type.\alpha_1\rightarrow\alpha_2\rightarrow
\tuple{\alpha_1}{\alpha_2}\\
\arg\times\arg & : &
\Pi\alpha_1,\alpha_2:\type.\set{\alpha_1}\rightarrow\set{\alpha_2}\rightarrow
\set{\tuple{\alpha_1}{\alpha_2}}\\
\mathbb{P}(\arg) & : &
\Pi\alpha:\type.\set{\alpha}\rightarrow\set{\set{\alpha}}\\
\arg=\arg & : & \Pi\alpha:\type.\alpha\rightarrow\alpha\rightarrow\omicron\\
\end{array}$
\end{flushleft}
\caption{Rewriting Rules of the Axiomatic Core of the \bmth{} Set Theory}
\label{fig:bset}
\end{figure}

To test \archsat{} in this theory, we consider 319~lemmas coming from Chap.~2 of
the \bbook{}~\cite{B-Book}. These lemmas are properties of various difficulty
regarding the set constructs introduced by the \bmth{} method. It should be
noted that these constructs and notations are, for a large part of them,
specific to the \bmth{} method, as they are used for the modeling of industrial
projects, and are not necessarily standard in set theory.

As tools, we actually consider two versions of \archsat{} in our test: one
version with rewriting and another one without rewriting (in this case, the
rules of the rewrite system are considered as usual equalities or
equivalences). We also include other automated theorem provers, able to deal
with first order logic with polymorphic types natively. In particular, we
consider \zenm{} (version~0.4.2), a tableau-based prover that is an extension of
\zenon{} to deduction modulo theory, as well as the \altergo{} SMT solver
(version~1.01). It would have been possible to also consider provers dealing
with pure first order logic and encode the polymorphic layer along the lines
of~\cite{BA13}. But preliminary tests have been conducted and very low results
have been obtained even for the best state-of-the-art provers (we have
considered \e{} and \cvc{} in particular), which tends to show that polymorphism
encoding adds a lot of noise in proof search and is not effective in practice.

The experiment was run on an \intel{}~3.50~GHz computer, with a timeout of 3~s
and a memory limit of 1~GiB. The results are summarized in Tab.~\ref{tab:bench}.

\setlength{\tabcolsep}{3pt}
\renewcommand{\arraystretch}{1.2}
\newcolumntype{C}{>{\centering}X}

\begin{table}[t]
\begin{center}
\begin{tabularx}{\textwidth}{|X|C|C|C|C|}
\hline
\begin{tabular}{l}
{\tiny \#Problems:}\\[-1mm]
319
\end{tabular} &
\begin{tabular}{c}
\archsat{}\\[-2mm]
{\tiny (w/ Rew)}
\end{tabular} &
\begin{tabular}{c}
\archsat{}\\[-2mm]
{\tiny (w/o Rew)}
\end{tabular} &
\begin{tabular}{c}
\zenm{}
\end{tabular} &
\begin{tabular}{c}
\altergo
\end{tabular}\tabularnewline
\hline{}
\begin{tabular}{l}
Proofs
\end{tabular} &
\begin{tabular}{c}
260
\end{tabular} &
\begin{tabular}{c}
-
\end{tabular} &
\begin{tabular}{c}
138
\end{tabular} &
\begin{tabular}{c}
232
\end{tabular}\tabularnewline
\hline{}
\begin{tabular}{l}
Rate
\end{tabular} &
\begin{tabular}{c}
81.5\%
\end{tabular} &
\begin{tabular}{c}
-
\end{tabular} &
\begin{tabular}{c}
43.2\%
\end{tabular} &
\begin{tabular}{c}
72.7\%
\end{tabular}\tabularnewline
\hline{}
\begin{tabular}{l}
Time \small{(s)}
\end{tabular} &
\begin{tabular}{c}
16.61
\end{tabular} &
\begin{tabular}{c}
-
\end{tabular} &
\begin{tabular}{c}
2.74
\end{tabular} &
\begin{tabular}{c}
8.31
\end{tabular}\tabularnewline
\hline{}
\begin{tabular}{l}
Unique
\end{tabular} &
\begin{tabular}{c}
0
\end{tabular} &
\begin{tabular}{c}
0
\end{tabular} &
\begin{tabular}{c}
0
\end{tabular} &
\begin{tabular}{c}
0
\end{tabular}\tabularnewline
\hline
\end{tabularx}
\end{center}
\caption{Experimental Results over the \bmth{} Set Theory Benchmark}
\label{tab:bench}
\end{table}

\renewcommand{\arraystretch}{1}
