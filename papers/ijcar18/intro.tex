% $Id$

\section{Introduction}

These days, Boolean satisfiability (SAT) solvers are undeniably quite reliable
tools despite the worst-case exponential run time of all known
algorithms. Modern SAT solvers provide a ``black-box'' procedure that can often
solve hard structured problems with over a million variables and several million
constraints. These tools are therefore used in many applications outside the
scope of knowledge representation and reasoning, which nonetheless was their
initial domain of application. The underlying representational formalism of SAT
solvers is propositional logic, and even if a given problem is not a pure
propositional problem, it is not unusual to encode it so that it can be viewed
as a propositional reasoning task. This encoding may be complex and possibly
costly, but the effectiveness of SAT solvers generally compensates for the cost
of this encoding.

This growing interest for SAT solvers has steered them toward a more first order
approach, with the possibility to deal with first order literals and reason
modulo theories. This resulted in the design of Satisfiability Modulo Theories
(SMT) solvers, which are built on top of SAT solvers. Over the last past few
years, SMT solvers have appeared as very efficient tools to reason over some
well identified theories (equality, uninterpreted functions, linear arithmetic,
arrays, etc.), and have allowed us to drag SAT solving toward first order logic.
Although modern SMT solvers support first order logic, most of them use
heuristic quantifier instantiation for incorporating quantifier reasoning with
ground decision procedures. This mechanism is relatively effective in some cases
in practice, but it is not refutationally complete for first order logic. Hints
(triggers) are usually required, and it is sensitive to the syntactic structure
of the formula, so that it fails to prove formulas that can be easily discharged
by provers based on more traditional first order proof search methods (tableaux,
resolution, etc.).

In this paper, we propose to overcome this problem of completeness for first
order logic in SMT solving by combining a SAT solver core with tableau
calculus. The tableau calculus is introduced as a regular theory of SMT solving,
and the rules are used to unfold propositional content into clauses while atomic
formulas are handled using satisfiability decision procedures as in
SMT solvers. To deal with quantified first order formulas, we use
metavariables and perform rigid unification modulo equalities and rewriting, for
which we introduce an algorithm based on superposition, but where all clauses
contain a single atomic formula.

We also propose to improve proof search by introducing rewriting into the SAT
solver as a regular SMT theory and along the lines of deduction modulo
theory. Deduction modulo theory~\cite{DA03} focuses on the computational part of
a theory, where axioms are transformed into rewrite rules, which induces a
congruence over propositions, and where reasoning is performed modulo this
congruence. In deduction modulo theory, this congruence is then induced by a set
of rewrite rules over both terms and propositions.

Our approach provides several advantages compared to usual SMT solving and first
order proof search methods. First, we benefit from the efficiency of a SAT
solver core together with a complete method of instantiation (when a
propositional model is found, we try to find a conflict between two literals by
unification). Second, it should be noted that our approach requires no change in
the architecture of the SMT solver, since the tableau calculus and rewriting are
seen as regular theories. Finally, no preliminary Skolemization and Conjunctive
Normal Form (CNF) transformation is required. This transformation is performed
lazily by applying the tableau rules progressively when a literal is propagated
or decided. This makes the production of genuine output proofs easier, contrary
to the usual approach, where the Skolemization/CNF translation is realized at
the beginning and externalized with respect to the proof search.

Our proposal combining SAT solving with tableau calculus and rewriting has been
implemented and the corresponding tool is called \archsat{}. This tool is able
to deal with first order logic extended to polymorphic types à la ML, through a
type system in the spirit of~\cite{BP13}. To test this tool, we propose a
benchmark in the framework of the set theory of the \bmth{}
method~\cite{B-Book}. This theory~\cite{BA15} has been expressed using first
order logic extended to polymorphic types and turned into a theory that is
compatible with deduction modulo theory, i.e. where a large part of axioms has
been turned into rewrite rules. The benchmark itself gathers 319~lemmas coming
from Chap.~2 of the \bbook{}~\cite{B-Book}.

The paper is organized as follows: in Sec.~\ref{sec:smt}, we recall the classic
architecture of SMT solvers, and we introduce the tableau and rewriting
theories; we then describe, in Sec.~\ref{sec:super}, our mechanism of equational
reasoning by means of rigid unit superposition; finally, in
Sec.~\ref{sec:bench}, we present some experimental results obtained by running
our implementation over a benchmark of problems in the \bmth{} set theory.
